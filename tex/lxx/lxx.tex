% LXX / Septuaginta Rahlfs 1935
\documentclass[10pt,oneside,footinclude=true,headinclude=true]{scrbook} % KOMA-Script book
\usepackage[paperheight=8.5in,paperwidth=5.5in,margin=0.3in,top=0.7in,bottom=0.5in]{geometry}
\usepackage[utf8]{inputenc}
\usepackage[english,greek]{babel}
\usepackage{setspace}
\usepackage{palatino}

\newcommand\en[1]{#1}

\begin{document}

\large
\setstretch{3}

%---------------------------------------------------------------------------------------
%	The text
%---------------------------------------------------------------------------------------

\foreignlanguage{english}{SEPTUAGINTA Rahlfs 1935}
\par
1:1 παροιμίαι Σολομώντος υιόυ Δαυίδ ος εβασίλευσεν εν Ισραήλ
2 γνώναι σοφίαν και παιδείαν νοήσαί τε λόγους φρονήσεως
3 δέξασθαί τε στροφάς λόγων και λύσεις αινιγμάτων νοήσαί τε δικαιοσύνην αληθή και κρίμα κατευθύνειν
4 ίνα δω ακάκοις πανουργίαν παιδί δε νέω αίσθησίν τε και έννοιαν
5 των δε γαρ ακούσας σοφός σοφώτερος έσται ο δε νοήμων κυβέρνησιν κτήσεται
6 νοήσει τε παραβολήν και σκοτεινόν λόγον ρήσεις τε σοφών και αινίγματα
7 αρχή σοφίας φόβος κυρίου σύνεσις δε αγαθή πάσι τοις ποιούσιν αυτήν ευσέβεια δε εις θεόν αρχή αισθήσεως σοφίαν δε και παιδείαν ασεβείς εξουθενήσουσιν
8 άκουε υιέ νόμους πατρός σου και μη απώση θεσμούς μητρός σου
9 στέφανον γαρ χαρίτων δέξη ση κορυφή και κλοιόν χρύσεον περί σω τραχήλω
10 υιέ μη σε πλανήσωσιν άνδρες ασεβείς μηδέ βουληθής
11 εάν παρακαλέσωσί σε λέγοντες ελθέ μεθ' ημών κοινώνησον αίματος κρύψωμεν δε εις γην άνδρα δίκαιον αδίκως
12 καταπίωμεν δε αυτόν ώσπερ άδης ζώντα και άρωμεν αυτού την μνήμην εκ γης
13 την κτήσιν αυτού την πολυτελή καταλαβώμεθα πλήσωμεν δε οίκους ημετέρους σκύλων
14 τον δε σον κλήρον βάλε εν ημίν και μαρσίππιον εν γενηθήτω ημίν
15 μη πορευθής οδούς μετ' αυτών έκκλινον τον πόδα σου εκ των τρίβων αυτών
16 οι γαρ πόδες αυτών εις κακίαν τρέχουσι και ταχινοί εισι του εκχέαι αίμα
17 ου γαρ αδίκως εκτείνονται δίκτυα πτερωτοίς
18 αυτοί γαρ οι φόνου μετέχοντες θησαυρίζουσιν εαυτοίς κακά
19 αύται αι οδοί εισι πάντων των συντελούντων τα άνομα τη γαρ ασεβεία την εαυτών ψυχήν αφαιρούνται
20 σοφία εν εξόδοις υμνείται εν δε πλατείαις παρρησίαν άγει
21 επ' άκρων δε τειχέων κηρύσσεται επί δε πύλαις πόλεως θαρρούσα λέγει
22 όσον αν χρόνον άκακοι έχωνται της δικαιοσύνης ουκ αισχυνθήσονται οι δε άφρονες της ύβρεως όντες επιθυμηταί ασεβείς γενόμενοι εμίσησαν αίσθησιν
23 και υπεύθυνοι εγένοντο ελέγχοις ιδού προήσομαι υμίν εμής πνοής ρήσιν διδάξω δε υμάς τον εμόν λόγον
24 επειδή εκάλουν και ουχ υπηκούσατε και εξέτεινον λόγους και ου προσείχετε
25 αλλά ακύρους εποιείτε εμάς βουλάς τοις δε εμοίς ελέγχοις ου προσείχετε
26 τοιγαρούν καγώ τη υμετέρα απωλεία επιγελάσομαι καταχαρούμαι δε ηνίκα αν έρχηται υμίν όλεθρος
27 και ως αν αφίκηται υμίν άφνω θόρυβος η δε καταστροφή ομοίως καταιγίδι παρή η όταν έρχηται υμίν θλίψις και πολιορκία
28 έσται γαρ όταν επικαλέσησθέ με εγώ δε ουκ εισακούσομαι υμών ζητήσουσί με κακοί και ουχ ευρήσουσιν
29 εμίσησαν γαρ σοφίαν τον δε φόβον κυρίου ου προείλαντο
30 ουδέ ήθελον εμαίς προσέχειν βουλαίς εμυκτήριζον δε εμούς ελέγχους
31 τοιγαρούν έδονται της εαυτών οδού τους καρπούς και της εαυτών ασεβείας πλησθήσονται
32 ανθ' ων γαρ ηδίκουν νηπίους φονευθήσονται και εξετασμός ασεβείς ολεί
33 ο δε εμού ακούων κατασκηνώσει επ' ελπίδι και ησυχάσει αφόβως από παντός κακού
\par
2:1 υιέ εάν δεξάμενος ρήσιν εμής εντολής κρύψης παρά σεαυτώ
2 υπακούσεται σοφίας το ους σου και παραβαλείς καρδίαν σου εις σύνεσιν παραβαλείς δε αυτήν επί νουθέτησιν τω υιώ σου
3 εάν γαρ την σοφίαν επικαλέση και τη συνέσει δως φωνήν σου
4 την δε αίσθησιν ζητήσης μεγάλη τη φωνή και εάν ζητήσης αυτήν ως αργύριον και ως θησαυρούς εξερευνήσης αυτήν
5 τότε συνήσεις φόβον κυρίου και επίγνωσιν θεού ευρήσεις
6 ότι κύριος δίδωσι σοφίαν και από προσώπου αυτού γνώσις και σύνεσις
7 και θησαυρίζει τοις κατορθούσι σωτηρίαν υπερασπιεί δε την πορείαν αυτών
8 του φυλάξαι οδόν δικαιωμάτων και οδόν ευλαβουμένων αυτόν διαφυλάξει
9 τότε συνήσεις δικαιοσύνην και κρίμα και κατορθώσεις πάντας άξονας αγαθούς
10 εάν γαρ έλθη η σοφία εις σην διάνοιαν η δε αίσθησις τη ση ψυχή καλή είναι δόξη
11 βουλή καλή φυλάξει σε έννοια δε οσία τηρήσει σε
12 ίνα ρύσηταί σε από οδού κακής και από ανδρός λαλούντος μηδέν πιστόν
13 ω οι εγκαταλείποντες οδούς ευθείας του πορεύεσθαι εν οδοίς σκότους
14 ω οι ευφραινόμενοι επί κακοίς και χαίροντες επί διαστροφή κακή
15 ων αι τρίβοι σκολιαί και καμπύλαι αι τροχιαί αυτών
16 του μακράν σε ποιήσαι από οδού ευθείας και αλλότριον της δικαίας γνώμης υιέ μη σε καταλάβη κακή βουλή
17 η απολιπούσα διδασκαλίαν νεότητος και διαθήκην θείαν επιλελησμένη
18 έθετο γαρ παρά τω θανάτω τον οίκον αυτής και παρά τω άδη μετά των γηγενών τους άξονας αυτής
19 πάντες οι παραπορευόμενοι εν αυτή ουκ αναστρέψουσιν ουδέ μη καταλάβωσι τρίβους ευθείας ου γαρ καταλαμβάνονται υπό ενιαυτών ζωής
20 ει γαρ επορεύοντο τρίβους αγαθάς εύροσαν αν τρίβους δικαιοσύνης λείας χρηστοί έσονται οικήτορες γης άκακοι δε υπολειφθήσονται εν αυτή
21 ότι ευθείς κατασκηνώσουσι γην και όσιοι υπολειφθήσονται εν αυτή
22 οδοί δε ασεβών εκ γης ολούνται οι δε παράνομοι εξωσθήσονται απ' αυτής
\par
3:1 υιέ εμών νομίμων μη επιλανθάνου τα δε ρήματά μου τηρείτω ση καρδία
2 μήκος γαρ βίου και έτη ζωής και ειρήνην προσθήσουσί σοι
3 ελεημοσύναι και πίστεις μη εκλειπέτωσαν σοι άφαψαι δε αυτάς επί σω τραχήλω γράψον αυτάς επί πλακός καρδίας σου
4 και ευρήσεις χάριν και προνού καλά ενώπιον κυρίου και ανθρώπων
5 ίσθι πεποιθώς εν όλη καρδία επί θεώ επί δε ση σοφία μη επαίρου
6 εν πάσαις οδοίς σου γνώριζε αυτήν ίνα ορθοτομή τας οδούς σου
7 μη ίσθι φρόνιμος παρά σεαυτώ φοβού δε τον θεόν και έκκλινε από παντός κακού
8 τότε ίασις έσται τω σώματί σου και επιμέλεια τοις οστέοις σου
9 τίμα τον κύριον από σων δικαίων πόνων και απάρχου αυτώ από σων καρπών δικαιοσύνης
10 ίνα πιμπλώνται τα ταμιεία σου πλησμονή σίτου οίνω δε αι ληνοί σου εκβλύζωσιν
11 υιέ μη ολιγώρει παιδείας κυρίου μηδέ εκλύου υπ' αυτού ελεγχόμενος
12 ον γαρ αγαπά κύριος παιδεύει μαστιγοί δε πάντα υιόν ον παραδέχεται
13 μακάριος άνθρωπος ος εύρε σοφίαν και θνητός ος είδε φρόνησιν
14 κρείσσον γαρ αυτήν εμπορεύεσθαι η χρυσίου και αργυρίου θησαυρούς
15 τιμιωτέρα δε εστι λίθων πολυτελών παν δε τίμιον ουκ άξιον αυτής εστι
16 μήκος γαρ βίου και έτη ζωής εν τη δεξιά αυτής εν δε τη αριστερά αυτής πλούτος και δόξα εκ του στόματος αυτής εκπορεύεται δικαιοσύνη νόμον δε και έλεον επί γλώσσης φορεί
17 αι οδοί αυτής οδοί καλαί και πάσαι αι τρίβοι αυτής εν ειρήνη
18 ξύλον ζωής εστι πάσι τοις αντεχομένοις αυτής και τοις επερειδομένοις επ' αυτήν ως επί κύριον ασφαλής
19 ο θεός τη σοφία εθεμελίωσε την γην ητοίμασε δε ουρανούς εν φρονήσει
20 εν αισθήσει αυτού άβυσσοι ερράγησαν νέφη δε ερρύησαν δρόσους
21 υιέ μη παραρρυής τήρησον δε εμήν βουλήν και έννοιαν
22 ίνα ζήση η ψυχή σου και χάρις η περί σω τραχήλω εσται δε ίασις ταις σαρξί σου και επιμέλεια τοις σοις οστέοις
23 ίνα πορεύη πεποιθώς εν ειρήνη πάσας τας οδούς σου ο δε πους σου ου μη προσκόψη
24 εάν γαρ κάθη άφοβος έση εάν δε καθεύδης ηδέως υπνώσεις
25 και ου φοβηθήση πτόησιν επελθούσαν ουδέ ορμάς ασεβών επερχομένας
26 ο γαρ κύριος έσται επί πασών οδών σου και ερείσει σον πόδα ίνα μη σαλευθής
27 μη απόσχη ευ ποιείν ενδεή ηνίκα αν έχη η χειρ σου βοηθείν
28 μη είπης επανελθών επάνηκε και αύριον δώσω δυνατού σου όντος ευ ποιείν ου γαρ οίδας τι τέξεται η επιούσα
29 μη τέκταινε επί σον φίλον κακά παροικούντα και πεποιθότα επί σοι
30 μη φιλεχθρήσης προς άνθρωπον μάτην ίνα μη εις σε εργάσηται κακόν
31 μη κτήση κακών ανδρών ονείδη μηδέ ζηλώσης τας οδούς αυτών
32 ακάθαρτος γαρ έναντι κυρίου πας παράνομος εν δε δικαίοις ου συνεδριάζει
33 κατάρα κυρίου εν οίκοις ασεβών επαύλεις δε δικαίων ευλογούνται
34 κύριος υπερηφάνοις αντιτάσσεται ταπεινοίς δε δίδωσι χάριν
35 δόξαν σοφοί κληρονομήσουσιν οι δε ασεβείς ύψωσαν ατιμίαν
\par
4:1 ακούσατε παίδες παιδείαν πατρός και προσέχετε γνώναι έννοιαν
2 δώρον γαρ αγαθόν δωρούμαι υμίν τον εμόν λόγον μη εγκαταλίπητε
3 υιός γαρ εγενόμην καγώ πατρί υπήκοος και αγαπώμενος εν προσώπω μητρός
4 οι έλεγον και εδίδασκόν με ερειδέτω ο ημέτερος λόγος εις σην καρδίαν φύλασσε εντολάς μη επιλάθη
5  κτήσαι σοφίαν κτήσαι σύνεσιν μη επιλάθη μηδέ παρίδης ρήσιν εμού στόματος μηδέ εκκλίνης από ρημάτων στόματός μου
6 μη εγκαταλίπης αυτήν και ανθέξεταί σου εράσθητι αυτής και τηρήσει σε
7 αρχή σοφίας κτήσαι σοφίαν και εν πάση κτήσει σου κτήσαι σύνεσιν
8 περιχαράκωσον αυτήν και υψώσει σε τίμησον αυτήν ίνα σε περιλάβη
9 ίνα δω τη ση κεφαλή στέφανον χαρίτων στεφάνω δε τρυφής υπερασπίση σε
10 άκουε υιέ και δέξαι εμούς λόγους και πληθυνθήσεταί σοι έτη ζωής σου ίνα σοι γένωνται πολλαί οδοί βίου
11 οδούς γαρ σοφίας διδάσκω σε εμβιβάζω δε σε τροχιαίς ορθαίς
12 εάν γαρ πορεύη ου συγκλεισθήσεταί σου τα διαβήματα εάν δε τρέχης ου κοπιάσεις
13 επιλαβού εμής παιδείας μη αφής αλλά φύλαξον αυτήν σεαυτώ εις ζωήν σου
14 οδούς ασεβών μη επέλθης μηδέ ζηλώσης οδούς παρανόμων
15 εν ω αν τόπω στρατοπεδεύσωσι μη επέλθης εκεί έκκλινον απ' αυτών και παράλλαξον
16 ου γαρ μη υπνώσωσιν εάν μη κακοποιήσωσιν αφήρηται ύπνος απ' αυτών και ου κοιμώνται
17 οίδε γαρ σιτούνται σίτα ασεβείας οίνω δε παρανόμω μεθύσκονται
18 αι δε οδοί των δικαίων ομοίως φωτί λάμπουσι προπορεύονται και φωτίζουσιν έως αν κατορθώση η ημέρα
19 αι δε οδοί των ασεβών σκοτειναί ουκ οίδασι πως προσκόπτουσιν
20 υιέ εμή ρήσει πρόσεχε τοις δε εμοίς λόγοις παράβαλλε σον ους
21 όπως μη εκλίπωσί σε αι πηγαί σου φύλασσε αυτάς εν ση καρδία
22 ζωή γαρ εστι πάσι τοις ευρίσκουσιν αυτάς και πάση σαρκί ίασις
23 πάση φυλακή τήρει σην καρδίαν εκ γαρ τούτων έξοδοι ζωής
24 περίελε σεαυτού σκολιόν στόμα και άδικα χείλη μακράν από σου άπωσαι
25 οι οφθαλμοί σου ορθά βλεπέτωσαν τα δε βλέφαρά σου νευέτω δίκαια
26 ορθάς τροχιάς ποίει σοις ποσί και τας οδούς σου κατεύθυνε
27 μη εκκλίνης εις τα δεξιά μηδέ εις τα αριστερά απόστρεψον δε σον πόδα από οδού κακής οδούς γαρ τας εκ δεξιών οίδεν ο θεός διεστραμμέναι δε εισιν αι εξ αριστερών αυτός δε ορθάς ποιήσει τας τροχιάς σου τας δε πορείας σου εν ειρήνη προάξει
\par
5:1 υιέ εμή σοφία πρόσεχε εμοίς δε λόγοις παράβαλλε σον ους
2 ίνα φυλάξης έννοιαν αγαθήν αίσθησιν δε εμών χειλέων εντέλλομαί σοι μη πρόσεχε φαύλη γυναικί
3 μέλι γαρ αποστάζει από χειλέων γυναικός πόρνης η προς καιρόν λιπαίνει σον φάρυγγα
4 ύστερον μέντοι πικρότερον χολής ευρήσεις και ηκονημένον μάλλον μαχαίρας διστόμου
5 της γαρ αφροσύνης οι πόδες κατάγουσι τους χρωμένους αυτή μετά θανάτου εις τον άδην τα δε ίχνη αυτής ουκ ερείδεται
6  οδούς γαρ ζωής ουκ επέρχεται σφαλεραί δε αι τροχιαί αυτής και ουκ εύγνωστοι
7 νυν ουν υιέ άκουέ μου και μη ακύρους ποιήσης εμούς λόγους
8 μακράν ποίησον απ' αυτής σην οδόν μη εγγίσης προς θύραις οίκων αυτής
9 ίνα μη πρόη άλλοις ζωήν σου και σον βίον ανελεήμοσιν
10 ίνα μη πλησθώσιν αλλότριοι σης ισχύος οι δε σοι πόνοι εις οίκους αλλοτρίους εισέλθωσι
11 και μεταμεληθήση επ' εσχάτων ηνίκα αν κατατριβώσι σάρκες σώματός σου
12 και ερείς πως εμίσησα παιδείαν και ελέγχους εξέκλινεν η καρδία μου
13 ουκ ήκουον φωνήν παιδεύοντός με και διδάσκοντός με ουδέ παρέβαλλον το ους μου
14 παρ' ολίγον εγενόμην εν παντί κακώ εν μέσω εκκλησίας και συναγωγής
15 υιέ πίνε ύδατα από σων αγγείων και από σων φρεάτων πηγής
16 υπερεκχείσθω σοι τα ύδατα εκ της σης πηγής εις δε σας πλατείας διαπορευέσθω τα σα ύδατα
17 έστω σοι μόνω υπάρχοντα και μηδείς αλλότριος μετασχέτω σοι
18 η πηγή σου του ύδατος έστω σοι ιδία και συνευφραίνου μετά γυναικός της εκ νεότητός σου
19 έλαφος φιλίας και πώλός σου χαρίτων ομιλείτω σοι η δε ιδία ηγείσθω σου και συνέστω σοι εν παντί καιρώ εν γαρ τη ταύτης φιλία συμπεριφερόμενος πολλοστός έση
20 μη πολύς ίσθι προς αλλοτρίαν μηδέ συνέχου αγκάλαις της μη ιδίας
21 ενώπιον γαρ εισι των του θεού οφθαλμών οδοί ανδρός εις δε πάσας τας τροχιάς αυτού σκοπεύει
22 παρανομίαι άνδρα αγρεύουσι σειραίς δε των εαυτού αμαρτιών έκαστος σφίγγεται
23 ούτος τελευτά μετά απαιδεύτων εκ δε πλήθους της εαυτού βιότητος εξερρίφη και απώλετο δι' αφροσύνην
\par
6:1 υιέ εάν εγγυήση σον φίλον παραδώσεις σην χείρα εχθρώ
2 παγίς γαρ ισχυρά ανδρί τα ίδια χείλη και αλίσκεται ρήμασιν ιδίου στόματος
3 ποίει υιέ α εγώ σοι εντέλλομαι και σώζου ήκεις γαρ εις χείρας κακών διά σον φίλον ίσθι μη εκλυόμενος παρόξυνε δε και τον φίλον σου ον εγεγγυήσω
4 μη δως ύπνον σοις όμμασι μηδέ επινυστάξης σοις βλεφάροις
5 ίνα σώζη ώσπερ δορκάς εκ βρόχων και ώσπερ όρνεον εκ παγίδος
6 ίθι προς τον μύρμηκα ω οκνηρέ και ζήλωσον ιδών τας οδούς αυτού και γενού εκείνου σοφώτερος
7 εκείνω γαρ γεωργίου μη υπάρχοντος μηδέ τον αναγκάζοντα έχων μηδέ υπό δεσπότην ων
8 ετοιμάζεται θέρους την τροφήν πολλήν τε εν τω αμητώ ποιείται την παράθεσιν
9 έως τίνος οκνηρέ κατάκεισαι πότε δε εξ ύπνου εγερθήση
10 ολίγον μεν υπνοίς ολίγον δε κάθησαι μικρόν δε νυστάζεις ολίγον δε εναγκαλίζη χερσί στήθη
11 ειτ' εμπαραγίνεταί σοι ώσπερ κακός οδοιπόρος η πενία και η ένδεια ώσπερ αγαθός δρομεύς
12 ανήρ άφρων και παράνομος πορεύσεται οδούς ουκ αγαθάς
13 ο δε αυτός εννεύει οφθαλμώ σημαίνει δε ποδί διδάσκει δε νεύμασι δακτύλων
14 διεστραμμένη δε καρδία τεκταίνεται κακά εν παντί καιρώ ο τοιούτος ταραχάς συνίστησι πόλει
15 διά τούτο εξαπίνης έρχεται η απώλεια αυτού διακοπή και συντριβή ανίατος
16 ότι χαίρει πάσιν οις μισεί ο κύριος συντρίβεται δε δι' ακαθαρσίαν ψυχής
17 οφθαλμός υβριστού γλώσσα άδικος χείρες εκχέουσαι αίμα δίκαιον
18 και καρδία τεκταινομένη λογισμούς κακούς και πόδες επισπεύδοντες κακοποιείν
19 εκκαίει ψευδή μάρτυς άδικος και επιπέμπει κρίσεις αναμέσον αδελφών
20 υιέ φύλασσε νόμους πατρός σου και μη απώση θεσμούς μητρός σου
21 άφαψαι δε αυτούς επί ση ψυχή διαπαντός και εγκλοίωσαι περί σω τραχήλω
22 ηνίκα αν περιπατής επάγου αυτήν και μετά σου έστω ως δ' αν καθεύδης φυλασσέτω σε ίνα εγειρομένω συλλαλή σοι
23 ότι λύχνος εντολή νόμου και φως και οδός ζωής και έλεγχος και παιδεία
24 του διαφυλάσσειν σε από γυναικός υπάνδρου και από διαβολής γλώσσης αλλοτρίας
25 υιέ μη σε νικήση κάλλους επιθυμία μηδέ αγρευθής σοις οφθαλμοίς μηδέ συναρπασθής από των αυτής βλεφάρων
26 τιμή γαρ πόρνης όση και ενός άρτου γυνή δε ανδρών τιμίας ψυχάς αγρεύει
27 αποδήσει τις πυρ εν κόλπω τα δε ιμάτια ου κατακαύσει
28 η περιπατήσει τις επ' ανθράκων πυρός τους δε πόδας ου κατακαύσει
29 ούτως ο εισελθών προς γυναίκα ύπανδρον ουκ αθωωθήσεται ουδέ πας ο απτόμενος αυτής
30 ου θαυμαστόν εάν αλώ τις κλέπτων κλέπτει γαρ ίνα εμπλήση ψυχήν πεινώσαν
31 εάν δε αλώ αποτίσει επταπλάσια και πάντα τα υπάρχοντα αυτού δους ρύσεταί εαυτόν
32 ο δε μοιχός δι' ένδειαν φρενών απώλειαν τη ψυχή αυτού περιποιείται
33 οδύνας τε και ατιμίας υποφέρει το δε όνειδος αυτού ουκ εξαλειφθήσεται εις τον αιώνα
34 μεστός γαρ ζήλου θυμός ανδρός αυτής ου φείσεται εν ημέρα κρίσεως
35 ουκ ανταλλάξεται ουδενός λύτρου την έχθραν ουδέ μη διαλυθή πολλών δώρων
\par
7:1 υιέ φύλασσε εμούς λόγους τας δε εμάς εντολάς κρύψον παρά σεαυτώ
2 υιέ τίμα τον κύριον και ισχύσεις πλην δε αυτού μη φοβού άλλον φύλαξον εμάς εντολάς και βιώσεις τους δε εμούς λόγους ώσπερ κόρας ομμάτων
3 περίθου αυτούς σοις δακτύλοις επίγραψον δε επί το πλάτος της καρδίας σου
4 είπον την σοφίαν σην αδελφήν είναι την δε φρόνησιν γνώριμον περιποίησαι σεαυτώ
5 ίνα σε τηρήση από γυναικός αλλοτρίας και πονηράς εάν σε λόγοις τοις προς χάριν εμβάλληται
6 από γαρ θυρίδος εκ του οίκου αυτής εις τας πλατείας παρακύπτουσα
7 ον αν ίδη των αφρόνων τέκνων νεανίαν ενδεή φρενών
8 παραπορευόμενον εν γωνία εν διόδοις οίκων αυτής λαλούντα
9 εν σκότει εσπερινώ ηνίκα αν ησυχία νυκτερινή η και γνοφώδης
10 η δε γυνή συναντά αυτώ είδος έχουσα πορνικόν η ποιεί νέων εξίπτασθαι καρδίας
11 ανεπτερωμένη δε εστι και άσωτος εν οίκω δε ουχ ησυχάζουσιν οι πόδες αυτής
12 χρόνον γαρ τινα έξω ρέμβεται χρόνον δε εν πλατείαις παρά πάσαν γωνίαν ενεδρεύει
13 είτα επιλαβομένη εφίλησεν αυτόν αναιδεί δε προσώπω προσείπεν αυτώ
14 θυσία ειρηνική μοι εστί σήμερον αποδίδωμι τας ευχάς μου
15 ένεκα τούτου εξήλθον εις συνάντησίν σου ποθούσα το σον πρόσωπον εύρηκά σε
16 κειρίαις τέτακα την κλίνην μου αμφιτάποις διέστρωσα τοις απ' Αιγύπτου
17 διέρραγκα την κοίτην μου κρόκω τον δε οίκόν μου κινναμώμω
18 ελθέ και απολαύσωμεν φιλίας έως όρθρου δεύρο και εγκυλισθώμεν έρωτι
19 ου γαρ πάρεστιν ο ανήρ μου εν οίκω πεπόρευται οδόν μακράν
20 ένδεσμον αργυρίου λαβών εν χερσίν αυτού δι' ημερών πολλών επανήξει εις τον οίκον αυτού
21 απεπλάνησε δε αυτόν πολλή ομιλία βρόχοις τε τοις από χειλέων εξώκειλεν αυτόν
22 ο δε επηκολούθησεν αυτή κεπφωθείς ωσπερ δε βους επί σφαγήν άγεται και ώσπερ κύων επί δεσμούς και ως έλαφος τοξεύματι
23 πεπληγώς εις το ήπαρ σπεύδει δε ώσπερ όρνεον εις παγίδα ουκ ειδώς ότι περί ψυχής τρέχει
24 νυν ουν υιέ άκουέ μου και πρόσεχε ρήμασι στόματός μου
25 μη εκκλινάτω εις τας οδούς αυτής η καρδία σου και μη πλανηθής εν ατραποίς αυτής
26 πολλούς γαρ τρώσασα καταβέβληκε και αναρίθμητοί εισιν ους πεφόνευκεν
27 οδοί άδου ο οίκος αυτής κατάγουσαι εις τα ταμιεία του θανάτου
\par
8:1 συ την σοφίαν κηρύξεις ίνα φρόνησίς σοι υπακούση
2 επί γαρ των υψηλών άκρων εστίν αναμέσον δε των τρίβων έστηκε
3 παρά γαρ πύλαις δυναστών παρεδρεύει εν δε εισόδοις υμνείται
4 υμάς ω άνθρωποι παρακαλώ και προϊεμαι εμήν φωνήν υιοίς ανθρώπων
5 νοήσατε άκακοι πανουργίαν οι δε απαίδευτοι ένθεσθε καρδίαν
6 εισακούσατέ μου σεμνά γαρ ερώ και ανοίσω από χειλέων ορθά
7 ότι αλήθειαν μελετήσει ο λάρυγξ μου εβδελυγμένα δε εναντίον εμού χείλη ψευδή
8 μετά δικαιοσύνης πάντα τα ρήματα του στόματός μου ουδέν εν αυτοίς σκολιόν ουδέ στραγγαλιώδες
9 πάντα ενώπια τοις συνιούσι και ορθά τοις ευρίσκουσι γνώσιν
10 λάβετε παιδείαν και μη αργύριον και γνώσιν υπέρ χρυσίον δεδοκιμασμένον
11  κρείσσων γαρ σοφία λίθων πολυτέλων παν δε τίμιον ουκ άξιον αυτής εστιν
12 εγώ η σοφία κατεσκήνωσα βουλήν και γνώσιν και έννοιαν εγώ επεκαλεσάμην
13 φόβος κυρίου μισεί αδικίαν ύβριν τε και υπερηφανίαν και οδούς πονηρών μεμίσηκα δε εγώ διεστραμμένας οδούς κακών
14 εμή βουλή και ασφάλεια εμή φρόνησις εμή δε ισχύς
15 δι' εμού βασιλείς βασιλεύουσι και οι δυνάσται γράφουσι δικαιοσύνην
16 δι' εμού μεγιστάνες μεγαλύνονται και τύραννοι δι' εμού κρατούσι γης
17 εγώ τους εμέ φιλούντας αγαπώ οι δε εμέ ζητούντες ευρήσουσι χάριν
18 πλούτος και δόξα εμοί υπάρχει και κτήσις πολλών και δικαιοσύνη
19 βέλτιον εμέ καρπίζεσθαι υπέρ χρυσίον και λίθον τίμιον τα δε εμά γεννήματα κρείσσον αργυρίου εκλεκτου
20 εν οδοίς δικαιοσύνης περιπατώ και αναμέσον τρίβων δικαιώματος αναστρέφομαι
21 ίνα μερίσω τοις εμέ αγαπώσιν ύπαρξιν και τους θησαυρούς αυτών εμπλήσω αγαθών εάν αναγγείλω υμίν τα καθ' ημέραν γινόμενα μνημονεύσω τα εξ αιώνος αριθμήσαι
22 κύριος έκτισέ με αρχήν οδών αυτού εις έργα αυτού
23 προ του αιώνος εθεμελίωσέ με εν αρχή προ του την γην ποιήσαι
24 και προ του τας αβύσσους ποιήσαι προ του προελθείν τας πηγάς των υδάτων
25 προ του όρη εδρασθήναι προ δε πάντων βουνών γεννά με
26 κύριος εποίησε χώρας και αοικήτους και άκρα οικούμενα της υπ' ουρανών
27 ηνίκα ητοίμαζε τον ουρανόν συμπαρήμην αυτώ και ότε αφώριζε τον εαυτού θρόνον επ' ανέμων
28 ηνίκα ισχυρά εποίει τα άνω νέφη και ως ασφαλείς ετίθει πηγάς της υπ' ουρανόν
29 εν τω τιθέναι τη θαλάσση ακριβασμόν αυτού και ύδατα ου παρελεύσονται στόματος αυτού και ως ισχυρά εποίει τα θεμέλια της γης
30 ήμην παρ' αυτώ αρμόζουσα εγώ ήμην η προσέχαιρε καθ' ημέραν δε ευφραινόμην εν προσώπω αυτού εν παντί καιρώ
31 ότε ενευφραίνετο την οικουμένην συντέλεσας και ενευφραίνετο εν υιοίς ανθρώπων
32 νυν ουν υιέ άκουέ μου και μακάριοι οι οδούς μου φυλάσσοντες
33 ακούσατε σοφίαν και σοφίσθητε και μη αποσφραγήτε μακάριος ανήρ ος εισακούσεταί μου και άνθρωπος ος τας εμάς οδούς φυλάξει
34 αγρυπνών επ' εμαίς θύραις καθ' ημέραν τηρών σταθμούς εμών εισόδων
35 αι γαρ έξοδοί μου έξοδοι ζωής και ετοιμάζεται θέλησις παρά κυρίου
36 οι δε αμαρτάνοντες εις εμέ ασεβούσιν εις τας εαυτών ψυχάς και οι μισούντές με αγαπώσι θάνατον
\par
9:1 η σοφία ωκοδόμησεν εαυτή οίκον και υπήρεισε στύλους επτά
2 έσφαξε τα εαυτής θύματα εκέρασεν εις κρατήρα τον εαυτής οίνον και ητοιμάσατο την εαυτής τράπεζαν
3 απέστειλε τους εαυτής δούλους συγκαλούσα μετά υψηλού κηρύγματος επί κρατήρα λέγουσα
4 ος εστιν άφρων εκκλινάτω προς με και τοις ενδεέσι φρενών είπεν
5 έλθατε φάγετε των εμών άρτών και πίετε οίνον ον εκέρασα υμίν
6 απολείπετε αφροσύνην και ζήσεσθε  και ζητήσατε φρόνησιν ίνα βιώσητε και κατορθώσατε εν γνώσει σύνεσιν
7 ο παιδεύων κακούς λήψεται εαυτώ ατιμίαν ελέγχων δε τον ασεβή μωμήσεται εαυτόν
8 μη έλεγχε κακους ίνα μη μισήσωσί σε έλεγχε σοφόν και αγαπήσει σε
9 δίδου σοφώ αφορμήν και σοφώτερος έσται γνώριζε δικαίω και προσθήσει του δέχεσθαι
10 αρχή σοφίας φόβος κυρίου και βουλή αγίων σύνεσις το γαρ γνώναι νόμον διανοίας εστίν αγαθής
11 τούτω γαρ τω τρόπω πολύν ζήσεις χρόνον και προστεθήσεταί σοι έτη ζωής
12 υιέ εάν σοφός γένη σεαυτώ σοφός έση  εάν δε κακός αποβής μόνος αν αντλήσεις κακά
13 γυνή άφρων και θρασεία ενδεής ψωμού γίνεται η ουκ επίσταται αισχύνην
14 εκάθισεν επί θύραις του εαυτής οίκου επί δίφρου εμφανώς εν πλατείαις
15 προσκαλουμένη τους παριόντας οδόν και κατευθύνοντας εν ταις οδοίς αυτών
16 ος εστιν υμών αφρονέστατος εκκλινάτω προς με και τοις ενδεέσι φρονήσεως παρακελεύομαι λέγουσα
17 άρτων κρυφίων ηδέως άψασθε και ύδατος κλοπής γλυκερού πίετε
18 ο δε ουκ οίδεν ότι γηγενείς παρ' αυτή όλλυνται και επί πέταυρον άδου συναντά
\par
10:1 υιός σοφός ευφραίνει πατέρα υιός δε άφρων λύπη τη μητρί
2 ουκ ωφελήσουσι θησαυροί ανόμους δικαιοσύνη δε ρύσεται εκ θανάτου
3 ου λιμοκτονήσει κύριος ψυχήν δικαίαν ζωήν δε ασεβών ανατρέψει
4 πενία άνδρα ταπεινοί χείρες δε ανδρείων πλουτίζουσιν υιός πεπαιδευμένος σοφός έσται τω δε άφρονι διακόνω χρήσεται
5 διεσώθη από καύματος υιός νοήμων ανεμόφθορος δε γίνεται εν αμητώ υιός παράνομος
6 ευλογιά κυρίου επί κεφαλήν δικαίου στόμα δε ασεβών καλύψει πένθος άωρον
7 μνήμη δικαίου μετ' εγκωμίου όνομα δε ασεβούς σβέννυται
8 σοφός καρδία δέξεται εντολάς ο δε άστεγος χείλεσι σκολιάζων υποσκελισθήσεται
9 ος πορεύεται απλώς πορεύεται πεποιθώς ο δε διαστρέφων τας οδούς αυτού γνωσθήσεται
10 ο εννεύων οφθαλμόν μετά δόλου συνάγει ανδράσι λύπας ο δε ελέγχων μετά παρρησίας ειρηνοποιεί
11 πηγή ζωής εν χειρί δικαίου στόμα δε ασεβούς καλύψει απώλεια
12 μίσος εγειρεί νείκος πάντας δε τους μη φιλονεικούντας καλύπψει φιλία
13 ος εκ χειλέων προφέρει σοφίαν ράβδω τύπτει άνδρα ακάρδιον
14 σοφοί κρύψουσιν αίσθησιν στόμα δε προπετούς εγγίζει συντριβή
15 κτήσις πλουσίων πόλις οχυρά συντριβή δε ασεβών πενία
16 έργα δικαίων ζωήν ποιεί καρπός δε ασεβών αμαρτίας
17 οδούς ζωής φυλάσσει παιδεία παιδεία δε ανεξέλεγκτος πλανάται
18 καλύπτουσιν έχθραν χείλη δίκαια οι δε εκφέροντες λοιδορίας αφρονέστατοί εισιν
19 εκ πολυλογίας ουκ εκφεύξη αμαρτίαν φειδόμενος δε χειλέων νοήμων έση
20 άργυρος πεπυρωμένος γλώσσα δικαίου καρδία δε ασεβούς εκλείψει
21 χείλη δικαίων επίσταται υψηλά οι δε άφρονες εν ενδεία τελευτώσιν
22 ευλογία κυρίου επί κεφαλήν δικαίου αύτη πλουτίζει και ου μη προστεθή αυτή λύπη εν καρδία
23 εν γέλωτι άφρων πράσσει κακά η δε σοφία ανδρί τίκτει φρόνησιν
24 εν απωλεία ασεβής περιφέρεται επιθυμία δε δικαίου δεκτή
25 παραπορευομένης καταιγίδος αφανίζεται ο ασεβής δίκαιος δε εκκλίνας σώζεται εις τον αιώνα
26 ώσπερ όμφαξ οδούσι βλαβερόν και καπνός όμμασιν ούτως παρανομία τοις χρωμένοις αυτή
27 φόβος κυρίου προστίθησιν ημέρας έτη δε ασεβών ολιγωθήσεται
28 εγχρονίζει δικαίοις ευφροσύνη ελπίς δε ασεβών όλλυται
29 οχύρωμα οσίου φόβος κυρίου συντριβή δε τοις εργαζομένοις κακά
30 δίκαιος εις τον αιώνα ουκ ενδώσει ασεβείς δε ουκ οικήσουσι γην
31 στόμα δικαίου αποστάζει σοφίαν γλώσσα δε αδίκου εξολείται
32 χείλη ανδρών δικαίων αποστάζει χάριτας στόμα δε ασεβών αποστρέφεται
\par
11:1 ζυγοί δόλιοι βδέλυγμα ενώπιον κυρίου στάθμιον δε δίκαιον δεκτόν αυτώ
2 ου εάν εισέλθη ύβρις εκεί και ατιμία στόμα δε ταπεινών μελετά σοφίαν
3  αποθανών δίκαιος έλιπε μετάμελον πρόχειρος δε γίνεται και επίχαρτος ασεβών απώλεια
4 τελειότης ευθειών οδηγήσει αυτούς και υποσκελισμός αθετούντων προνομεύσει αυτούς ουκ ωφελήσει υπάρχοντα εν ημέρα θυμού και δικαιοσύνη ρύσεται από θανάτου
5 δικαιοσύνη αμώμου ορθοτομει οδούς ασέβεια δε περιπίπτει αδικία
6 δικαιοσύνη ανδρών ορθών ρύσεται αυτούς τη δε αβουλία αλίσκονται παράνομοι
7 τελευτήσαντος ανδρός δικαίου ουκ όλλυται ελπίς το δε καύχημα των ασεβών όλλυται
8 δίκαιος εκ θήρας εκδύνει αντ' αυτού δε παραδίδοται ο ασεβής
9 εν στόματι ασεβών παγίς πολίταις αίσθησις δε δικαίων εύοδος
10 εν αγαθοίς δικαίων κατώρθωσε πόλις και εν απωλεία ασεβών αγαλλίαμα
11 εν ευλογία ευθείων υψωθήσεται πόλις στόμασι δε ασεβών κατεσκαφήσεται
12 μυκτηρίζει πολίτας ενδεής φρενών ανήρ δε φρόνιμος ησυχίαν άγει
13 ανήρ δίγλωσσος αποκαλύπτει βουλάς εν συνεδρίω πιστός δε πνοή κρύπτει πράγματα
14 οις μη υπάρχει κυβέρνησις πίπτουσιν ώσπερ φύλλα σωτηρία δε υπάρχει εν πολλή βουλή
15 πονηρός κακοποιεί όταν συμμίξη δικαίω μισεί δε ήχον ασφαλείας
16 γυνή ευχάριστος εγείρει ανδρί δόξαν θρόνος δε ατιμίας γυνή μισούσα δίκαια πλούτου οκνηροί ενδεείς γίνονται οι δε ανδρείοι ερείδονται πλούτω
17 τη ψυχή αυτού αγαθόν ποιεί ανήρ ελεήμων εξολλύει δε αυτού σώμα ο ανελεήμων
18 ασεβής ποιεί έργα άδικα σπέρμα δε δικαίων μισθός αληθείας
19 υιός δίκαιος γεννάται εις ζωήν διωγμός δε ασεβούς εις θάνατον
20 βδέλυγμα κυρίω διεστραμμέναι οδοί προσδεκτοί δε αυτώ πάντες άμωμοι εν οδώ
21 χειρί χείρας εμβαλών αδίκως ουκ ατιμώρητος έσται κακών ο δε σπείρων δικαιοσύνην λήψεται μισθόν πιστόν
22 ώσπερ ενώτιον χρυσούν εν ρινί υός ούτω γυναικί κακόφρονι κάλλος
23 επιθυμία δικαίων πάσα αγαθή ελπίς δε ασεβών απολείται
24 εισίν οι τα ίδια σπείροντες πλείονα ποιούσιν εισί δε και οι συνάγοντες ελαττονούνται
25 ψυχή ευλογουμένη πάσα απλή ανήρ δε θυμώδης ουκ ευσχήμων
26 ο συνέχων σίτον υπολείποιτο αυτόν τοις έθνεσιν ευλογία δε εις κεφαλήν του μεταδιδόντος
27 τεκταινόμενος αγαθά ζητεί χάριν αγαθήν εκζητούντα δε κακά καταλήψεται αυτόν
28 ο πεποιθώς επί πλούτω εαυτού ούτος πεσείται ο δε αντιλαμβανόμενος δικαίων ανατελεί
29 ο μη συμπεριφερόμενος τω εαυτού οίκω κληρονομήσει άνεμον δουλεύσει δε άφρων φρονίμω
30 εκ καρπού δικαιοσύνης φύεται δένδρον ζωής αφαιρούνται δε άωροι ψυχαί παρανόμων
31 ει ο μεν δίκαιος μόλις σώζεται ο ασεβής και αμαρτωλός που φανείται
\par
12:1 ο αγαπών παιδείαν αγαπά αίσθησιν ο δε μισών ελέγχους άφρων
2 κρείσσων ο ευρών χάριν παρά κυρίω ανήρ δε παράνομος παρασιωπηθήσεται
3 ου κατορθώσει άνθρωπος εξ ανόμου αι δε ρίζαι των δικαίων ουκ εξαρθήσονται
4 γυνή ανδρεία στέφανος τω ανδρί αυτής ώσπερ δε εν ξύλω σκώληξ ούτως άνδρα απόλλυσι γυνή κακοποιός
5 λογισμοί δικαίων κρίματα κυβερνώσι δε ασεβείς δόλους
6 λόγοι ασεβών δόλιοι εις αίμα στόμα δε ορθών ρύσεται αυτούς
7 ου εάν στραφή ο ασεβής αφανίζεται οίκοι δε δικαίων παραμενούσι
8 στόμα συνετού εγκωμιάζεται υπό ανδρός νωθροκάρδιος δε μυκτηρίζεται
9 κρείσσων ανήρ εν ατιμία δουλεύων εαυτώ η τιμήν εαυτώ περιτιθείς και προσδεόμενος άρτου
10 δίκαιος οικτείρει ψυχάς κτηνών αυτού τα δε σπλάγχνα των ασεβών ανελεήμονα
11 ο εργαζόμενος την εαυτού γην εμπλησθήσεται άρτων οι δε διώκοντες μάταια ενδεείς φρενών ος εστιν ηδύς εν οίνων διατριβαίς εν τοις εαυτού οχυρώμασι καταλείψει ατιμίαν
12 επιθυμίαι ασεβών κακαί αι δε ρίζαι των ευσεβών εν οχυρώμασι
13 διά αμαρτίαν χειλέων εμπίπτει εις παγίδας αμαρτωλός εκφεύγει δε εξ αυτών δίκαιος
14 από καρπών στόματος ψυχή ανδρός πλησθήσεται αγαθών ανταπόδομα δε χειλέων αυτού αποδοθήσεται αυτώ
15 οδοί αφρόνων ορθαί ενώπιον αυτών εισακούει δε συμβουλίας σοφός
16 άφρων αυθήμερον εξαγγέλλει οργήν αυτού κρύπτει δε την εαυτού ατιμίαν πανούργος
17 επιδεικνυμένην πίστιν απαγγέλλει δίκαιος ο δε μάρτυς των αδίκων δόλιος
18 εισίν οι λέγοντες τιτρώσκουσιν ως μάχαιρα γλώσσαι δε σοφών ιώνται
19 χείλη αληθινά κατορθοί μαρτυρίαν μάρτυς δε ταχύς γλώσσαν έχει άδικον
20 δόλος εν καρδία τεκταινομένου κακά οι δε βουλόμενοι ειρήνην ευφρανθήσονται
21 ουκ αρέσει τω δικαίω ουδέν άδικον οι δε ασεβείς πλησθήσονται κακών
22 βδέλυγμα κυρίω χείλη ψευδή ο δε ποιών πίστεις δεκτός παρ' αυτώ
23 ανήρ συνετός θρόνος αισθήσεως καρδία δε αφρόνων συναντήσεται αραίς
24 χειρ εκλεκτών κρατήσει ευχερώς δόλιοι δε έσονται εις προνομήν
25 φοβερός λόγος καρδίαν ταράσσει ανδρός αγγελία δε αγαθή ευφραίνει αυτόν
26 επιγνώμων δίκαιος εαυτού φίλος έσται  αμαρτάνοντας δε καταδιώξεται κακά η δε οδός των ασεβών πλανήσει αυτούς
27 ουκ επιτεύξεται δόλιος θήρας κτήμα δε τίμιον ανήρ καθαρός
28 εν οδοίς δικαιοσύνης ζωή οδοί δε μνησικάκων εις θάνατον
\par
13:1 υιός πανούργος υπήκοος πατρί υιός δε ανήκοος εν απωλεία
2 από καρπών δικαιοσύνης φάγεται αγαθός ψυχαί δε παρανόμων ολούνται άωροι
3 ος φυλάσσει το εαυτού στόμα τηρεί την εαυτού ψυχήν ο δε προπετής χείλεσι πτοήσει εαυτόν
4 εν επιθυμίαις εστί πας αεργός χείρες δε ανδρείων εν επιμελεία
5 λόγον άδικον μισεί δίκαιος ασεβής δε αισχύνεται και ουκ έξει παρρησίαν
6 δικαιοσύνη φυλάσσει ακάκους οδώ τους δε ασεβείς φαύλους ποιεί αμαρτία
7 εισίν οι πλουτίζοντες εαυτούς μηδέν έχοντες και εισίν οι ταπεινούντες εαυτούς εν πολλώ πλούτω
8 λύτρον ανδρός ψυχής ο ίδιος πλούτος πτωχός δε ουχ υφίσταται απειλήν
9 φως δικαίοις διαπαντός φως ασεβών σβέννυται
10 κακός μεθ' ύβρεως πράσσει κακά οι δε αυτών επιγνώμονες σοφοί
11 ύπαρξις επισπουδαζομένη μετά ανομίας ελάσσων γίνεται ο δε συνάγων εαυτώ μετ' ευσεβείας πληθυνθήσεται δίκαιος οικτείρει και κιχρά
12  κρείσσων εναρχόμενος βοηθών καρδία του επαγγελλομένου και εις ελπίδα άγοντος δένδρον γαρ ζωής επιθυμία αγαθή
13 ος καταφρονεί πράγματος καταφρονηθήσεται εξ αυτού ο δε φοβούμενος εντολήν ούτος υγιαίνει υιώ δολίω ουδέν έσται αγαθόν οικέτη δε σοφώ εύοδοι έσονται πράξεις και κατευθυνθήσεται η οδός αυτού
14 νόμος σοφού πηγή ζωής ο δε ανούς υπό παγίδος θανείται
15 σύνεσις αγαθή δίδωσι χάριν το δε γνώναι νόμον διανοίας εστίν αγαθής οδοί δε καταφρονούντων εν απωλεία
16 πας πανούργος πράσσει μετά γνώσεως ο δε άφρων εξεπέτασεν εαυτού κακίαν
17 βασιλεύς θρασύς πεσείται εις κακά άγγελος δε σοφός ρύσεται αυτόν
18 πενίαν και ατιμίαν αφαιρείται παιδεία ο δε φυλάσσων ελέγχους δοξασθήσεται
19 επιθυμίαι ασεβών ηδύνουσι ψυχήν έργα δε ασεβών μακράν από γνώσεως
20 συμπορευόμενος σοφοίς σοφος εση ο δε συμπορευόμενος άφροσι γνωσθήσεται
21 αμαρτάνοντας καταδιώξεται κακά τους δε δικαίους καταλήψεται αγαθά
22 αγαθός ανήρ κληρονομήσει υιούς υιών θησαυρίζεται δε δικαίοις πλούτος ασεβών
23 δίκαιοι ποιήσουσιν εν πλούτω έτη πολλά άδικοι δε απολούνται συντόμως
24 ος φείδεται της βακτηρίας μισεί τον υιόν αυτού ο δε αγαπών επιμελώς παιδεύει
25 δίκαιος έσθων εμπιπλάται την ψυχήν αυτού ψυχαί δε ασεβών ενδεείς
\par
14:1 σοφαί γυναίκες ωκοδόμησαν οίκους η δε άφρων κατέσκαψε ταις χερσίν αυτής
2 ο πορευόμενος ορθώς φοβείται τον κύριον ο δε σκολιάζων ταις οδοίς αυτού ατιμασθήσεται
3 εκ στόματος αφρόνων βακτηρία ύβρεως χείλη δε σοφών φυλάσσει αυτούς
4 ου μη εισί βόες φάτναι καθαραί ου δε πολλά γεννήματα φανερά βοός ισχύς
5 μάρτυς πιστός ου ψεύδεται εκκαίει δε ψευδή μάρτυς άδικος
6 ζητήσεις σοφίαν παρά κακοίς και ουχ ευρήσεις αίσθησις δε παρά φρονίμοις ευχερής
7 πάντα εναντία ανδρί άφρονι όπλα δε αισθήσεως χείλη σοφά
8 σοφία πανούργων επιγνώσεται τας οδούς αυτών άνοια δε αφρόνων εν πλάνη
9 οικίαι παρανόμων οφειλήσουσι καθαρισμόν οικίαι δε δικαίων δεκταί
10 καρδία ανδρός αισθητική λυπηρά ψυχή αυτού όταν δε ευφραίνηται ουκ επιμίγνυται ύβρει
11 οικίαι ασεβών αφανισθήσονται σκηναί δε κατορθούντων στήσονται
12 έστιν οδός η δοκεί ορθή είναι παρά ανθρώποις τα δε τελευταία αυτής έρχεται εις πυθμένα άδου
13 εν ευφροσύναις ου προσμίγνυται λύπη τελευταία δε χαράς εις πένθος έρχεται
14 των εαυτού οδών πλησθήσεται θρασυκάρδιος από δε των διανοημάτων αυτού ανήρ αγαθός
15 άκακος πιστεύει παντί λόγω πανούργος δε έρχεται εις μετάνοιαν
16 σοφός φοβηθείς εξέκλινεν από κακού ο δε άφρων εαυτώ πεποιθώς μίγνυται ανόμω
17 οξύθυμος πράσσει μετά αβουλίας ανήρ δε φρόνιμος πολλά υποφέρει
18 μεριούνται άφρονες κακίαν οι δε πανούργοι κρατήσουσιν αισθήσεως
19 ολισθήσουσι κακοί έναντι αγαθών και ασεβείς θεραπεύσουσι θύρας δικαίων
20 φίλοι μισήσουσι φίλους πτωχούς φίλοι δε πλουσίων πολλοί
21 ο ατιμάζων πένητας αμαρτάνει ελεών δε πτωχούς μακαριστός
22 πλανώμενοι τεκταίνουσι κακά έλεον δε και αλήθειαν τεκταίνουσιν αγαθοί ουκ επίστανται έλεον και πίστιν τέκτονες κακών ελεημοσύναι δε και πίστεις παρά τέκτοσιν αγαθοίς
23 εν παντί μεριμνώντι ένεστι περισσόν ο δε ηδύς και ανάλγητος εν ενδεία έσται
24 στέφανος σοφών πλούτος αυτών η δε διατριβή αφρόνων κακή
25 ρύσεται κακών ψυχήν μάρτυς πιστός εκκαίει δε ψευδή δόλιος
26 εν φόβω κυρίου ελπίς ισχύος τοις δε τέκνοις αυτού καταλείπει έρεισμα
27 πρόσταγμα κυρίου πηγή ζωής ποιεί δε εκκλίνειν εκ παγίδος θανάτου
28 εν πολλώ έθνει δόξα βασιλέως εν δε εκλείψει λαού συντριβή δυνάστου
29 μακρόθυμος ανήρ πολύς εν φρονήσει ο δε ολιγόψυχος ισχυρώς άφρων
30 πραύθυμος ανήρ καρδίας ιατρός σης δε οστέων καρδία αισθητική
31 ο συκοφαντών πένητα παροξύνει τον ποιήσαντα αυτόν ο δε τιμών αυτόν ελεεί πτωχόν
32 εν κακία αυτού απωσθήσεται ασεβής ο πεποιθώς τη εαυτού οσιότητι δίκαιος
33 εν καρδία αγαθή ανδρός αναπαύσεται σοφία εν δε καρδία αφρόνων ου διαγινώσκεται
34 δικαιοσύνη υψοί έθνος ελασσονούσι δε φυλάς αμαρτίαι
35 δεκτός βασιλεί υπηρέτης νοήμων τη δε εαυτού ευστροφία αφαιρείται ατιμίαν
\par
15:1 οργή απόλλυσι και φρονίμους απόκρισις δε υποπίπτουσα αποστρέφει θυμόν λόγος δε λυπηρός εγείρει οργάς
2 γλώσσα σοφών καλά επίσταται στόμα δε αφρόνων αναγγέλλει κακά
3 εν παντί τόπω οφθαλμοί κυρίου σκοπεύουσι κακούς τε και αγαθούς
4 ίασις γλώσσης δένδρον ζωής ο δε συντηρών αυτήν πλησθήσεται πνεύματος
5 άφρων μυκτηρίζει παιδείαν πατρός ο δε φυλάσσων εντολάς πανουργότερος εν πλεοναζούση δικαιοσύνη ισχύς πολλή οι δε ασεβείς ολόρριζοι εκ γης απολούνται
6 οίκοις δικαίων ισχύς πολλή καρποί δε ασεβών ολούνται
7 χείλη σοφών δέδεται αισθήσει καρδίαι δε αφρόνων ουκ ασφαλείς
8 θυσίαι ασεβών βδέλυγμα κυρίω ευχαί δε κατευθυνόντων δεκταί παρ' αυτώ
9 βδέλυγμα κυρίω οδοί ασεβούς διώκοντας δικαιοσύνην αγαπά
10 παιδεία ακάκου γνωρίζεται υπό των παριόντων οι δε μισούντες ελέγχους τελευτώσιν αισχρώς
11 άδης και απώλεια φανερά παρά κυρίω πως ουχί και αι καρδίαι των ανθρώπων
12 ουκ αγαπήσει απαίδευτος τους ελέγχοντας αυτόν μετά δε σοφών ουχ ομιλήσει
13 καρδίας ευφραινομένης πρόσωπον θάλλει εν δε λύπαις ούσης σκυθρωπάζει
14 καρδία ορθή ζητεί αίσθησιν στόμα δε απαιδεύτων γνώσεται κακά
15 πάντα τον χρόνον οι οφθαλμοί των κακών προσδέχονται κακά οι δε αγαθοί ησυχάζουσι διαπαντός
16 κρείσσον μικρά μερίς μετά φόβου κυρίου η θησαυροί μεγάλοι μετά αφοβίας
17 κρείσσων ξενισμός λαχάνων προς φιλίαν και χάριν η παράθεσις μόσχων μετά έχθρας
18 ανήρ θυμώδης παρασκευάζει μάχας μακρόθυμος δε και την μέλλουσαν καταπραύνει μακρόθυμος ανήρ κατασβέσει κρίσεις ο δε ασεβής εγείρει μάλλον
19 οδοί αεργών εστρωμέναι ακάνθαις αι δε των ανδρείων τετριμμέναι
20 υιός σοφός ευφραίνει πατέρα υιός δε άφρων μυκτηρίζει μητέρα αυτού
21 ανοήτου τρίβοι ενδεείς φρενών ανήρ δε φρόνιμος κατευθύνων πορεύεται
22 υπερτίθενται λογισμούς οι μη τιμώντες συνέδρια εν δε καρδίαις βουλευομένων μένει βουλή
23 ου μη υπακούση ο κακός αυτή ουδε μη είπη καίριόν τι και καλόν τω κοινώ
24 οδοί ζωής διανοήματα συνετού ίνα εκκλίνας εκ του άδου σωθή
25 οίκους υβριστών κατασπά κύριος εστήρισε δε όριον χήρας
26 βδέλυγμα κυρίω λογισμός άδικος αγνών δε ρήσεις σεμναί
27 εξόλλυσιν εαυτόν ο δωρολήπτης ο δε μισών δώρων λήψεις σώζεται ελεημοσύναις και πίστεσιν αποκαθαίρονται αμαρτίαι τω δε φόβω κυρίου εκκλίνει πας από κακού
28 καρδίαι δικαίων μελετώσι πίστεις στόμα δε ασεβών αποκρίνεται κακά δεκταί παρά κυρίω οδοί ανδρών δικαίων διά δε αυτών και οι εχθροί φίλοι γίνονται
29 μακράν απέχει ο θεός από ασεβών ευχαίς δε δικαίων επακούει κρείσσων ολίγη λήψις μετά δικαιοσύνης η πολλά γεννήματα μετά αδικίας
30 θεωρών οφθαλμός καλά ευφραίνει καρδίαν φήμη δε αγαθή πιαίνει οστά
31 ο εισάκουων ελέγχους ζωής εν μεσω σοφών αυλισθησέται
32 ος απωθείται παιδείαν μισεί εαυτόν ο δε τηρών ελέγχους αγαπά ψυχήν αυτού
33 φόβος κυρίου παιδεία και σοφία και αρχή δόξης αποκριθήσεται αυτή προπορεύεται δε ταπεινοίς δόξα
\par
16:1  καρδία ανδρός λογιζέσθω δίκαια ίνα υπό του θεού διορθωθή τα διαβήματα αυτού πάντα τα έργα του ταπεινού φανερά παρά τω θεώ οι δε ασεβείς εν ημέρα κακή ολούνται
5  ακάθαρτος παρά θεώ πας υψηλοκάρδιος χειρί δε χείρας εμβαλών αδίκως ουκ αθωωθήσεται αρχή οδού αγαθής το ποιείν τα δίκαια δεκτά δε παρά θεώ μάλλον η θύειν θυσίας ο ζητών τον κύριον ευρήσει γνώσιν μετά δικαιοσύνης οι δε ορθώς ζητούντες αυτόν ευρήσουσιν ειρήνην πάντα τα έργα του κυρίου μετά δικαιοσύνης φυλάσσεται δε ο ασεβής εις ημέραν κακήν
10 μαντείον επί χείλεσι βασιλέως εν δε κρίσει ου μη πλανηθή το στόμα αυτού
11 ροπή ζυγού δικαιοσύνη παρά θεώ τα δε έργα αυτού στάθμια δίκαια
12 βδέλυγμα βασιλεί ο ποιών κακά μετά γαρ δικαιοσύνης ετοιμάζεται θρόνος αρχής
13 δεκτά βασιλεί χείλη δίκαια λόγους δε ορθούς αγαπά ο κύριος
14 θυμός βασιλέως άγγελος θανάτου ανήρ δε σοφός εξιλάσεται αυτόν
15 εν φωτί ζωής υιός βασιλέως οι δε προσδεκτοί αυτώ ώσπερ νέφος όψιμον
16 νοσσιαί σοφίας αιρετώτεραι χρυσίου νοσσιαί δε φρονήσεως αιρετώτεραι υπέρ αργύριον
17 τρίβοι ζωής εκκλίνουσιν από κακών μήκος δε βίου οδοί δικαιοσύνης ο δεχόμενος παιδείαν εν αγαθοίς έσται ο δε φυλάσσων ελέγχους σοφισθήσεται ος φυλάσσει τας εαυτού οδούς τηρεί την εαυτού ψυχήν αγαπών δε ζωήν αυτού φείσεται στόματος αυτού
18 προ συντριβής ηγείται ύβρις προ δε πτώματος κακοφροσύνη
19 κρείσσων πραύθυμος μετά ταπεινώσεως η ος διαιρείται σκύλα μετά υβριστών
20 συνετός εν πράγμασιν ευρετής αγαθών πεποιθώς δε επί θεώ μακαριστός
21 τους σοφούς και συνετούς φαύλους καλούσιν οι δε γλυκείς εν λόγω πλείονα ακούσονται
22 πηγή ζωής έννοια τοις κεκτημένοις παιδεία δε αφρόνων κακή
23 καρδία σοφού νοήσει τα από του ιδίου στόματος επί δε χείλεσι φορέσει επιγνωμοσύνην
24 κηρία μέλιτος λόγοι καλοί γλύκασμα δε αυτών ίασις ψυχής
25 εισίν οδοί δοκούσαι ορθαί είναι ανδρί τα μέντοι τελευταία αυτών βλέπει εις πυθμένα άδου
26 ανήρ εν πόνοις πονεί εαυτώ και εκβιάζεται αυτού την απώλειαν
27 ο μέντοι σκολιός επί τω εαυτού στόματι φορεί την απώλειαν ανήρ άφρων ορύσσει εαυτώ κακά επί δε των εαυτού χειλέων θησαυρίζει πυρ
28 ανήρ σκολιός διαπέμπεται κακά και λαμπτήρα δόλου πυρσεύσει κακοίς και διαχωρίζει φίλους
29 ανήρ παράνομος αποπειράται φίλων και απάγει αυτούς οδούς ουκ αγαθάς
30 στηρίζων οφθαλμούς αυτού λογίζεται διεστραμμένα ορίζει δε τοις χείλεσιν αυτού πάντα τα κακά ούτος κάμινός εστι κακίας
31 στέφανος καυχήσεως γήρας εν δε οδοίς δικαιοσύνης ευρίσκεται
32 κρείσσων ανήρ μακρόθυμος ισχυρού και ανήρ φρόνησιν έχων γεωργίου μεγάλου ο δε κρατών οργής κρείσσων καταλαμβανομένου πόλιν
33 εις κόλπους επέρχεται πάντα τοις αδικοις παρά δε κυρίου πάντα τα δίκαια
\par
17:1 κρείσσων ψωμός μεθ' ηδονής εν ειρήνη η οίκος πλήρης πολλών αγαθών και αδίκων θυμάτων μετά μάχης
2 οικέτης νοήμων κρατήσει δεσποτών αφρόνων εν δε αδελφοίς διελείται μέρη
3 ώσπερ δοκιμάζεται εν καμίνω άργυρος και χρυσός ούτως εκλεκταί καρδίαι παρά κυρίω
4 κακός υπακούει γλώσσης παρανόμων άδικος δε προσέχει χείλεσι ψεύδεσιν
5 ο καταγελών πτωχού παροξύνει τον ποιήσαντα αυτόν ο δε επιχαίρων απολλυμένω ουκ αθωωθήσεται ο δε επισπλαγχνιζόμενος ελεηθήσεται
6 στέφανος γερόντων τέκνα τέκνων καύχημα δε τέκνων πατέρες αυτών του πιστού όλος ο κόσμος των χρημάτων του δε απίστου ουδέ οβολός
7 ουχ αρμόσει άφρονι χείλη πιστά ουδέ δικαίω χείλη ψευδή
8 μισθός χαρίτων παιδεία τοις χρωμένοις ου δ' αν επιστρέψη ευοδωθήσεται
9 ος κρύπτει αδικήματα ζητεί φιλίαν ος δε μισεί κρύπτειν διϊστησι φίλους και οικείους
10 συντρίβει απειλή καρδίαν φρονίμου άφρων δε μαστιγωθείς ουκ αισθάνεται
11 αντιλογίας εγείρει πας κακός ο δε κύριος άγγελον ανελεήμονα εκπέμψει αυτώ
12 εμπεσείται μέριμνα ανδρί νοήμονι οι δε άφρονες διαλογιούνται κακά
13 ος αποδίδωσι κακά αντί αγαθών ου κινηθήσεται κακά εκ του οίκου αυτού
14 εξουσίαν δίδωσι λόγοις αρχή δικαιοσύνης προηγείται δε της ενδείας στάσις και μάχη
15 ος δίκαιον κρίνει τον άδικον άδικον δε τον δίκαιον ακάθαρτος και βδελυκτός παρά θεώ
16 ινατί υπήρξε χρήματα άφρονι κτήσασθαι γαρ σοφίαν ακάρδιος ου δυνήσεται ος υψηλόν ποιεί τον εαυτού οίκον ζητεί συντριβήν ο δε σκολιάζων του μαθείν εμπεσείται εις κακά
17 εις πάντα καιρόν ο φίλος υπαρχέτω σοι αδελφοί εν ανάγκαις χρήσιμοι έστωσαν τούτου γαρ χάριν γεννώνται
18 ανήρ άφρων επικροτεί και επιχαίρει εαυτώ ως και ο εγγυώμενος εγγύη τον εαυτού φίλον
19 φιλαμαρτήμων χαίρει μάχαις
20 ο δε σκληροκάρδιος ου συναντά αγαθοίς ανήρ ευμετάβολος γλώσση εμπεσείται εις κακά
21 καρδία δε άφρονος οδύνη τω κεκτημένω αυτήν ουκ ευφραίνεται πατήρ εφ' υιώ απαιδεύτω υιός δε φρόνιμος ευφραίνει μητέρα αυτού
22 καρδία ευφραινομένη ευεκτείν ποιεί ανδρός δε λυπηρού ξηραίνεται τα οστά
23 λαμβάνοντος δώρα εν κόλποις αδίκως ου κατευοδούνται αι οδοί ασεβής δε εκκλίνει οδούς δικαιοσύνης
24 πρόσωπον συνετόν ανδρός σοφού οι δε οφθαλμοί του άφρονος επ' άκρα γης
25 οργή πατρί υιός άφρων και οδύνη τη τεκούση αυτού
26 ζημιούν άνδρα δίκαιον ου καλόν ουδέ όσιον επιβούλευειν δυνάσταις δικαίοις
27 ος φείδεται ρήμα προέσθαι σκληρόν επιγνώμων μακρόθυμος δε ανήρ φρόνιμος
28 ανοήτω επερωτήσαντι σοφία λογισθήσεται ενεόν δε τις εαυτόν ποιήσας δόξει φρόνιμος είναι
\par
18:1 προφάσεις ζητεί ανήρ βουλόμενος χωρίζεσθαι από φίλων εν παντί δε καιρώ επονείδιστος έσται
2 ου χρείαν έχει σοφίας ενδεής φρενών μάλλον γαρ άγεται αφροσύνη
3 όταν έλθη ασεβής εις βάθος κακών καταφρονεί επέρχεται δε αυτώ ατιμία και όνειδος
4 ύδωρ βαθύ λόγος εν καρδία ανδρός ποταμός δε αναπηδύει και πηγή ζωής
5 θαυμάσαι πρόσωπον ασεβούς ου καλόν ουδέ όσιον εκκλίνειν το δίκαιον εν κρίσει
6 χείλη άφρονος άγουσιν αυτόν εις κακά το δε στόμα αυτού το θρασύ θάνατον επικαλείται
7 στόμα άφρονος συντριβή αυτώ τα δε χείλη αυτού παγίς τη ψυχή αυτού
8 οκνηρούς καταβάλλει φόβος ψυχαί δε ανδρογύνων πεινάσουσιν
9 ο μη ιώμενος εαυτόν εν τοις έργοις αυτού αδελφός εστι του λυμαινομένου εαυτόν
10 εκ μεγαλωσύνης ισχύος όνομα κυρίου αυτώ δε προσδραμόντες δίκαιοι υψούνται
11 ύπαρξις πλουσίου ανδρός πόλις οχυρά η δε δόξα αυτής μέγα επισκιάζει
12 προ συντριβής υψούται καρδία ανδρός και προ δόξης ταπεινούται
13 ος αποκρίνεται λόγον πριν ακούσαι αφροσύνη αυτώ εστι και όνειδος
14 θυμόν ανδρός πρα=νει θεράπων φρόνιμος ολιγόψυχον δε άνδρα τις υποίσει
15 καρδία φρονίμου κτάται αίσθησιν ώτα δε σοφών ζητεί έννοιαν
16 δόμα ανθρώπου εμπλατύνει αυτόν και παρά δυνάσταις καθιζάνει αυτόν
17 δίκαιος εαυτού κατήγορος εν πρωτολογία ως δ' αν επιβάλη ο αντίδικος ελέγχεται
18 αντιλογίας παύει κλήρος εν δε δυνάσταις ορίζει
19 αδελφός υπό αδελφού βοηθούμενος ως πόλις οχυρά και υψηλή ισχύει δε ώσπερ τεθεμελιωμένον βασίλειον
20 από καρπών στόματος ανήρ πίμπλησι κοιλίαν αυτού από δε καρπών χειλέων αυτού εμπλησθήσεται
21 θάνατος και ζωή εν χειρί γλώσσης οι δε κρατούντες αυτής έδονται τους καρπούς αυτής
22 ος εύρε γυναίκα αγαθήν εύρε χάριτας έλαβε δε παρά κυρίου ιλαρότητα ος εκβάλλει γυναίκα αγαθήν εκβάλλει τα αγαθά ο δε κατέχων μοιχαλίδα άφρων και ασεβής
23 δεήσεις φθέγγεται πένης ο δε πλούσιος αποκρίνεται σκληρά
24 ανήρ εταίρων προς εταιρίαν και έστι φίλος προσκολληθείς υπέρ αδελφόν
\par
19:1 κρείσσων εστί πτωχός πορευόμενος εν απλότητι αυτού η στρεβλός τοις χείλεσιν αυτού και αυτός ανόητος
2 και γε χωρίς επιστήμης ψυχή ουκ αγαθή και ο σπεύδων τοις ποσίν αμαρτάνει
3 αφροσύνη ανδρός λυμαίνεται τας οδούς αυτού τον δε θεόν αιτιάται τη καρδία αυτού
4 πλούτος προστίθησι φίλους πολλούς ο δε πτωχός και από του υπάρχοντος φίλου λείπεται
5 μάρτυς ψευδής ουκ ατιμώρητος έσται ο δε εγκαλών αδίκως ου διαφεύξεται
6 πολλοί θεραπεύουσι πρόσωπα βασιλέως πας δε ο κακός γίνεται όνειδος ανδρί
7 πας ος αδελφόν πτωχόν μισεί και φιλίας μακράν έσται έννοια αγαθή τοις ειδόσιν αυτήν εγγιεί ανήρ δε φρόνιμος ευρήσει αυτήν ο πολλά κακοποιών τελεσιουργεί κακίας ος δε ερεθίζει λόγοις ου σωθήσεται
8 ο κτώμενος φρόνησιν αγαπά εαυτόν ος δε φυλάσσει φρόνησιν ευρήσει αγαθά
9 μάρτυς ψευδής ουκ ατιμώρητος έσται ος δ' αν εκκαύσει κακίαν απολείται υπ' αυτής
10 ου συμφέρει άφρονι τρυφή και εάν οικέτης άρξηται μεθ' ύβρεως δυναστεύειν
11 ελεήμων ανήρ μακροθυμεί το δε καύχημα αυτού επέρχεται παρανόμοις
12 βασιλέως απειλή ομοία βρυγμώ λέοντος ώσπερ δε δρόσος επί χόρτω ούτως το ιλαρόν αυτού
13 αισχύνη πατρί υιός άφρων και ουχ αγναί ευχαί από μισθώματος εταίρας
14 οίκον και ύπαρξιν μερίζουσι πατέρες παισί παρά δε κυρίου αρμόζεται γυνή ανδρί
15 δειλία κατέχει ανδρόγυνον ψυχή δε αεργού πεινάσει
16 ος φυλάσσει εντολήν τηρεί την εαυτού ψυχήν ο δε καταφρονών των εαυτού οδών απολείται
17 δανείζει θεώ ο ελεών πτωχόν κατά δε το δόμα αυτού ανταποδώσει αυτώ
18 παίδευε υιόν σου ούτω γαρ έσται εύελπις εις δε ύβριν μη επαίρου τη ψυχή
19 κακόφρων ανήρ πολλά ζημιωθήσεται εάν δε λοιμεύηται και την ψυχήν αυτού προσθήσει
20 άκουε υιέ παιδείαν πατρός σου ίνα σοφός γένη επ' εσχάτων σου
21 πολλοί λογισμοί εν καρδία ανδρός η δε βουλή του κυρίου εις τον αιώνα μένει
22 καρπός ανδρί ελεημοσύνη κρείσσων δε πτωχός δίκαιος η πλούσιος ψεύστης
23 φόβος κυρίου εις ζωήν ανδρί ο δε άφοβος αυλισθήσεται εν τόποις ου ουκ επισκοπείται γνώσις
24 ο εγκρύπτων εις τον κόλπον αυτού χείρας αδίκως ουδέ τω στόματι ου μη προσαγάγη αυτάς
25 λοιμού μαστιγουμένου άφρων πανουργότερος έσται εάν δε ελέγχης άνδρα φρόνιμον νοήσει αίσθησιν
26 ο ατιμάζων πατέρα και απωθούμενος μητέρα αυτού καταισχυνθήσεται και επονείδιστος έσται
27 υιός απολειπόμενος φυλάξαι παιδείαν πατρός μελετήσει ρήσεις κακάς
28 ο εγγυώμενος παίδα άφρονα καθυβρίσει δικαίωμα στόμα δε ασεβών καταπίεται κρίσεις
29 ετοιμάζονται ακολάστοις μάστιγες και τιμωρίαι ομοίως άφροσιν
\par
20:1 ακόλαστον οίνος και υβριστικόν μέθη πας δε ο λυμαινόμενος ουκ έσται σοφός
2 ου διαφέρει απειλή βασιλέως θυμού λέοντος ο δε παροξύνων αυτόν και επιμιγνύμενος αμαρτάνει εις την εαυτού ψυχήν
3 δόξα ανδρί αποστρέφεσθαι λοιδορίας πας δε άφρων τοιούτοις συμπλέκεται
4 ονειδιζόμενος οκνηρός ουκ αισχύνεται ωσαύτως και ο δανειζόμενος σίτον εν αμητώ
5 ύδωρ βαθύ βουλή εν καρδία ανδρός ανήρ δε φρόνιμος εξαντλήσει αυτήν
6 μέγα άνθρωπος και τίμιον ανήρ ελεήμων άνδρα δε πιστόν έργον ευρείν
7 ος αναστρέφεται άμωμος εν δικαιοσύνη μακαρίους τους παίδας αυτού καταλείψει
8 όταν βασιλεύς δίκαιος καθίση επί θρόνου ουκ εναντιούται εν οφθαλμοίς αυτού παν πονηρόν
9 τις καυχήσεται αγνήν έχειν την καρδίαν η τις παρρήσιασεται καθαρός είναι από αμαρτιών
10 στάθμιον μέγα και μικρόν και μέτρα δισσά ακάθαρτα ενώπιον κυρίου και αμφότερα και ο ποιών αυτά
11 εν τοις επιτηδεύμασι αυτού συμποδισθήσεται νεανίσκος μετά οσίου και ευθεία η οδός αυτού
12 ους ακούει και οφθαλμός ορά κυρίου έργα και αμφότερα
13 μη αγάπα καταλαλείν ίνα μη εξαρθής διάνοιξον δε τους οφθαλμούς σου και εμπλήσθητι άρτων
14 κακόν κακόν λέγει ο αγοράζων και ως απέλθη τότε καυχήσεται
15 εστι χρυσός και πλήθος λίθων πολυτελών και σκεύη έντιμα χείλη συνέσεως
16 αφελού το ιμάτιον του εγγυώντος αντί αλλοτρίου και αντί αλλοτρίας λάβε ενέχυρον παρ' αυτού
17 ηδύς ανθρώπω άρτος ψεύδους και έπειτα εμπλησθήσεται το στόμα αυτού χαλίκων
18 διαλογισμοί εν βουλή στερεώνται κυβερνήσεσι δε γίνεται πόλεμος
19 ο ανακαλύπτων βουλάς εν συνεδρίω πορεύεται δίγλωσσος και μετά πλαντύνοντος τα εαυτού μη μίχθητι χείλη
20 κακολογούντος πατέρα η μητέρα σβεσθήσεται λαμπτήρ αι δε κόραι των οφθαλμών αυτού όψονται σκότος
21 μερίς επισπουδαζομένη εν πρώτοις εν τοις τελευταίοις ουκ ευλογηθήσεται
22 μη είπης τίσομαι τον εχθρόν αλλ' υπόμεινον τον κύριον ίνα σοι βοηθήση
23 βδέλυγμα κυρίω δίσσον στάθμιον και ζυγός δόλιος ου καλόν ενώπιον αυτού
24 παρά κυρίου ευθύνεται τα διαβήματα ανδρί θνητός δε πως αν νοήσαι τας οδούς αυτού
25 παγίς ανδρί ταχύ τι των ιδίων αγιάσαι μετά γαρ το εύξασθαι μετανοείν γίνεται
26 λικμήτωρ ασεβών βασιλεύς σοφός και επιβαλεί αυτοίς τροχόν
27 φως κυρίου πνοή ανθρώπων ος ερευνά ταμιεία κοιλίας
28 ελεημοσύνη και αλήθεια φυλακή βασιλεί και περικυκλώσουσιν εν δικαιοσύνη τον θρόνον αυτού
29 κόσμος νεανίαις σοφία δόξα δε πρεσβυτέρων πολιαί
30 υπώπια και συντρίμματα συναντά κακοίς πληγαί δε εις ταμιεία κοιλίας
\par
21:1 ώσπερ ορμή ύδατος ούτως καρδία βασιλέως εν χειρί θεού ου εάν θέλων νεύση εκεί έκλινεν αυτήν
2 πας ανήρ φαίνεται εαυτώ δίκαιος κατευθύνει δε καρδίας κύριος
3 ποιείν δίκαια και αληθεύειν αρεστά παρά θεώ μάλλον η θυσιών αίμα
4 μεγαλόφρων εφ' ύβρει θρασυκάρδιος λαμπτήρ δε ασεβών αμαρτία
5 διαλογισμοί ανδρείου εις πλησμονήν και πας ο σπεύδων εν ελάσσονι
6 ο ενεργών θησαυρίσματα γλώσση ψευδεί μάταια διώκει και έρχεται επί παγίδος θανάτου
7 όλεθρος ασεβέσιν επιξενωθήσεται ου γαρ βούλονται πράσσειν τα δίκαια
8 προς τους σκολιούς σκολιάς οδούς αποστέλλει ο θεός αγνά γαρ και ορθά τα έργα αυτού
9 κρείσσον οικείν επί γωνίας υπαίθρου η κεκονιαμένοις μετά αδικίας και εν οίκω κοινώ
10 ψυχή ασεβούς επιθυμεί κακών ουκ ελεηθήσεται υπ' ουδενός των ανθρώπων
11 ζημιουμένου ακολάστου πανουργότερος γίνεται ο άκακος συνιών δε σοφός δέξεται γνώσιν
12 συνιεί δίκαιος καρδίας ασεβών και φαυλίζει ασεβείς εν κακοίς
13 ος φράσσει τα ώτα αυτού του μη ακούσαι ασθενούς και αυτός επικαλέσεται και ουκ έσται ο εισακούων
14 δόσις λάθριος ανατρέπει οργάς δώρων δε ο φειδόμενος θυμόν εγερεί ισχυρόν
15 ευφροσύνη δικαίων ποιείν κρίμα όσιος δε ακάθαρτος παρά κακούργοις
16 ανήρ πλανώμενος εξ οδού δικαιοσύνης εν συναγωγή γιγάντων αναπαύσεται
17 ανήρ ενδεής αγαπά ευφροσύνην φιλών οίνον και έλαιον εις πλούτον
18 περικάθαρμα δε δικαίου άνομος και αντί ευθέων παράνομος
19 κρείσσον οικείν εν τη ερήμω η μετά γυναικός μαχίμου και γλωσσώδους και οργίλου
20 θησαυρός επιθυμητός αναπαύσεται επί στόματος σοφού άφρονες δε άνδρες καταπίονται αυτόν
21 οδός δικαιοσύνης και ελεημοσύνης ευρήσει ζωήν και δόξαν
22 πόλεις οχυράς επέβη σοφός και καθείλε το οχύρωμα εφ' ω επεποίθεισαν οι ασεβείς
23 ος φυλάσσει το στόμα αυτού και την γλώσσαν διατηρεί εκ θλίψεως την ψυχήν αυτού
24 θρασύς και αυθάδης και αλαζών λοιμός καλείται ος δε μνησικακεί παράνομος
25 επιθυμίαι οκνηρόν αποκτεινούσιν ου γαρ προαιρούνται αι χείρες αυτού ποιείν τι
26 ασεβής επιθυμεί όλην την ημέραν επιθυμίας κακάς ο δε δίκαιος ελεεί και οικτείρει αφειδώς
27 θυσίαι ασεβών βδέλυγμα κυρίω και γαρ παρανόμως προσφέρουσιν αυτάς
28 μάρτυς ψευδής απολείται ανήρ υπήκοος φυλασσόμενος λαλήσει
29 ασεβής ανήρ αναιδώς υφίσταται προσώπω ο δε ευθύς αυτός συνιεί τας οδούς αυτού
30 ουκ έστι σοφία ουκ έστιν ανδρεία ουκ έστι βουλή προς τον ασεβή
31 ίππος ετοιμάζεται εις ημέραν πολέμου παρά δε κυρίου η βοήθεια
\par
22:1 αιρετώτερον όνομα καλόν η πλούτος πολύς υπέρ δε αργύριον και χρυσίον χάρις αγαθή
2 πλούσιος και πτωχός συνήντησαν αλλήλοις αμφοτέρους δε ο κύριος εποίησε
3 πανούργος ιδών πονηρόν τιμωρούμενον κραταιώς αυτός παιδεύεται οι δε άφρονες παρελθόντες εζημιώθησαν
4 γενεά σοφίας φόβος κυρίου και πλούτος και δόξα και ζωή
5 τρίβολοι και παγίδες εν οδοίς σκολιαίς ο δε φυλάσσων την εαυτού ψυχήν αφέξεται αυτών
6 εγκαίνισον το παιδίον κατά την οδόν αυτού και γε εάν γηράση ουκ αποστήσεται απ' αυτής
7 πλούσιοι πτωχών άρξουσι και οικέται ιδίοις δεσπόταις δανιούσιν
8 ο σπείρων φαύλα θερίσει κακά πληγήν δε έργων αυτού συντελέσει άνδρα ιλαρόν και δότην αγαπά ο θεός ματαιότητα δε έργων αυτού συντελέσει
9 ο ελεών πτωχόν αυτός διατραφήσεται των γαρ εαυτού άρτων δέδωκε τω πτωχώ νίκην και τιμήν περιποιείται ο δώρα δους την μέντοι ψυχήν αφαιρείται των κεκτημένων
10 έκβαλε εκ συνεδρίου λοιμόν και συνεξελεύσεται αυτώ νείκος όταν γαρ καθίση εν συνεδρίω πάντας ατιμάζει
11 αγαπά κύριος οσίας καρδίας δεκτοί δε αυτώ πάντες άμωμοι εν ταις οδοίς αυτών χείλεσι ποιμαίνει βασιλεύς
12 οι δε οφθαλμοί κυρίου διατηρούσιν αίσθησιν φαυλίζει δε λόγους παράνομος
13 προφασίζεται και λέγει οκνηρός λέων εν ταις οδοίς εν δε ταις πλατείαις φονευταί
14 βόθρος βαθύς στόμα παρανόμου ο δε μισηθείς υπό κυρίου εμπεσείται εις αυτόν
15 άνοια εξήπται καρδίας νέου ράβδος δε και παιδεία μακράν απ' αυτού
16 ο συκοφαντών πένητα πολλά ποιεί τα εαυτού κακά δίδωσι δε πλουσίω επ' ελάσσονι
17 λόγοις σοφών παράβαλλε το ους σου και άκουε εμών λόγων την δε σην καρδίαν επίστησον ίνα γνως
18 ότι καλοί εισι και εάν εμβάλης αυτούς εις την καρδίαν σου ευφρανούσί σε άμα επί σοις χείλεσιν
19 ίνα σου γένηται επί κύριον η ελπίς και γνωρίση σοι την οδόν αυτού
20 και συ δε απόγραψαι αυτά σεαυτώ τρισσώς εις βουλήν και γνώσιν
21 διδάσκω ουν σε αληθή λόγον και γνώσιν αγαθήν υπακούειν του αποκρίνεσθαί σε λόγους αληθείας τοις προβαλλομένοις σοι
22 μη αποβιάζου πένητα πτωχός γαρ εστι και μη ατιμάσης ασθενή εν πύλαις
23 ο γαρ κύριος κρινεί αυτού την δίκην και ρύση σην άσυλον ψυχήν
24 μη ίσθι εταίρος ανδρί θυμώδει φίλω δε οργίλω μη συναυλίζου
25 μήποτε μάθης των οδών αυτού και λάβης βρόχους τη ση ψυχή
26 μη δίδου σεαυτόν εις εγγυήν αισχυνόμενος πρόσωπον
27 εάν γαρ μη έχης πόθεν αποτίσης λήψονται το στρώμα το υπό τας πλευράς σου
28 μη μέταιρε όρια αιώνια α έστησαν οι πατέρες σου
29 ορατικόν άνδρα και οξύν εν τοις έργοις αυτού βασιλεύσι δει παρεστάναι και μη παρεστάναι ανδράσι νωθροίς
\par
23:1 εάν καθίσης δειπνήσαι επί τραπέζης δυνάστου νοητώς νόει τα παρατιθέμενά σοι
2 και επίβαλλε την χείρά σου ειδώς ότι τοιαύτά σε δει παρασκευάσαι ει δε απληστότερος ει
3 μη επιθύμει των εδεσμάτων αυτού ταύτα γαρ έχεται ζωής ψευδούς
4 μη παρεκτείνου πένης ων πλουσίω τη δε ση έννοια απόσχου
5 εάν επιστήσης το σον όμμα προς αυτόν ουδαμού φανείται κατεσκεύασται γαρ αυτώ πτέρυγες ώσπερ αετού και υποστρέφει εις τον οίκον του προεστηκότος αυτού
6 μη συνδείπνει ανδρί βασκάνω μηδέ επιθύμει των βρωμάτων αυτού
7 ον τρόπον γαρ ει τις καταπίοι τρίχα ούτως εσθίει και πίνει μηδέ προς σε εισαγάγης αυτόν και φάγης τον ψωμόν σου μετ' αυτού
8 εξεμέσει γαρ αυτόν και λυμανείται τους λόγους σου τους καλούς
9 εις ώτα άφρονος μηδέν λέγε μή ποτε μυκτηρίση τους συνετούς σου λόγους
10 μη μεταθής όρια αιώνια εις δε κτήμα ορφανών μη εισέλθης
11 ο γαρ λυτρούμενος αυτούς κύριος κραταιός εστι και κρινεί την κρίσιν αυτών μετά σου
12 δος εις παιδείαν την καρδίαν σου τα δε ωτά σου ετοίμασον λόγοις αισθήσεως
13 μη απόσχη νήπιον παιδεύειν ότι εάν πατάξης αυτόν ράβδω ου μη αποθάνη
14 συ μεν γαρ πατάξεις αυτόν ράβδω την δε ψυχήν αυτού εκ θανάτου ρύση
15 υιέ εάν σοφή γένηταί σου η καρδία ευφρανείς και την εμήν καρδίαν
16 και ενδιατρίψει λόγοις τα σα χείλη προς τα εμά χείλη εάν ορθά ώσι
17 μη ζηλούτω η καρδία σου αμαρτωλούς αλλά εν φόβω κυρίου ίσθι όλην την ημέραν
18 εάν γαρ τηρήσης αυτά έσται σοι έκγονα η δε ελπίς σου ουκ αποστήσεται
19 άκουε υιέ και σοφός γίνου και κατεύθυνε εννοίας σης καρδίας
20 μη ίσθι οινοπότης μηδέ εκτείνου συμβολαίς κρεών τε αγορασμοίς
21 πας γαρ μέθυσος και πορνοκόπος πτωχεύσει και ενδύσεται διερρηγμένα και ρακώδη πας υπνώδης
22 άκουε υιέ πατρός του γεννήσαντός σε και μη καταφρόνει ότι γεγήρακέ σου η μήτηρ
23 αλήθειαν κτήσαι και μη απώση σοφίαν και παιδείαν και σύνεσιν
24 καλώς εκτρέφει πατήρ δίκαιος επί δε υιώ σοφώ ευφραίνεται η ψυχή αυτού
25 ευφραινέσθω ο πατήρ και η μήτηρ επί σοι και χαιρέτω η τεκούσά σε
26 δος μοι υιέ σην καρδίαν οι δε σοι οφθαλμοί εμάς οδούς τηρείτωσαν
27 πίθος γαρ τετριμένος εστίν αλλότριος οίκος και φρεάρ στενόν αλλότριον
28 ούτος γαρ συντόμως απολείται και πας παράνομος αναλωθήσεται
29 τίνι ουαί τίνι θόρυβος τίνι κρίσεις τίνι αηδίαι και λέσχαι τίνι συντρίμματα διακενής τίνος πελιδνοί οι οφθαλμοί
30 ου των εγχρονιζόντων εν οίνοις ου των ιχνευόντων που πότοι γίνονται μη μεθύσκεσθε οίνω αλλά ομιλείτε ανθρώποις δικαίοις και ομιλείτε εν περιπάτοις
31 εάν γαρ εις τας φιάλας και τα ποτήρια δως τους οφθαλμούς σου ύστερον περιπατήσεις γυμνότερος υπέρου
32 το δε έσχατον ώσπερ υπό όφεως πεπληγώς εκτείνεται και ώσπερ υπό κεράστου διαχείται αυτού ο ιός
33 οι οφθαλμοί σου όταν ίδωσιν αλλοτρίαν το στόμα σου τότε λαλήσει σκολιά
34 και κατακείση ώσπερ εν καρδία θαλάσσης και ώσπερ κυβερνήτης εν πολλώ κλύδωνι
35 ερείς δε έτυπτόν με και ουκ επόνεσα και ενέπαιξάν μοι εγώ δε ουκ ήδειν πότε όρθρος έσται ίνα ελθών ζητήσω μεθ' ων συνελεύσομαι
\par
24:1 υιέ μη ζηλώσης κακούς άνδρας μηδέ επιθυμήσης είναι μετ' αυτών
2 ψευδή γαρ μελετά η καρδία αυτών και πόνους τα χείλη αυτών λαλεί
3 μετά σοφίας οικοδομείται οίκος και μετά συνέσεως ανορθούται
4 μετά αισθήσεως εμπίμπλανται τα ταμιεία παντός πλούτου τιμίου και καλού
5 κρείσσων σοφός ισχυρού και ανήρ φρόνησιν έχων γεωργίου μεγάλου
6 μετά κυβερνήσεως γίνεται πόλεμος βοήθεια δε μετά καρδίας βουλευτικής
7 σοφία και έννοια αγαθή εν πύλαις σοφών σοφοί ουκ εκκλίνουσιν εκ νόμου κυρίου αλλά λογίζονται εν συνεδρίοις απαιδεύτοις συναντά θάνατος
9  αποθνήσκει δε άφρων εν αμαρτίαις ακαθαρσία ανδρί λοιμώ
10 εμμολυνθήσεται εν ημέρα κακή και εν ημέρα θλίψεως έως αν εκλείπη
11 ρύσαι αγομένους εις θάνατον και εκπρίου κτεινομένους μη φείση
12 εάν δε είπης ουκ οίδα τούτον γίνωσκε ότι κύριος καρδίας πάντων γινώσκει και ο πλάσας πνοήν πάσιν αυτός οίδε πάντα ος αποδίδωσιν εκάστω κατά τα έργα αυτού
13 φάγε μέλι υιέ αγαθόν γαρ κηρίον ίνα γλυκανθή σου ο φάρυγξ
14 ούτως γαρ αισθήση σοφίαν τη ση ψυχή εάν γαρ εύρης έσται καλή η τελευτή σου και ελπίς σε ουκ καταλείψει
15 μη προσαγάγης ασεβή νομή δικαίων μηδέ απατηθής χορτασία κοιλίας
16 επτάκις γαρ πεσείται ο δίκαιος και αναστήσεται οι δε ασεβείς ασθενήσουσιν εν κακοίς
17 εάν πέση ο εχθρός σου μη επιχαρής αυτώ εν δε τω υποσκελίσματι αυτού μη επαίρου
18 ότι όψεται κύριος και ουκ αρέσει αυτώ και αποστρέψει τον θυμόν αυτού απ αυτού
19 μη χαίρε επί κακοποιοίς μηδέ ζήλου αμαρτωλούς
20 ου γαρ μη γένηται έκγονα πονηρών λαμπτήρ δε ασεβών σβεσθήσεται
21 φοβού τον θεόν υιέ και βασιλέα και μηθ ετέρω αυτών απειθήσης
22 εξαίφνης γαρ τίσονται τους ασεβείς τας δε τιμωρίας αμφοτέρων τις γνώσεται
23 ταύτα δε λέγω υμίν τοις σοφοίς επιγινώσκειν αιδείσθαι πρόσωπον εν κρίσει ου καλόν
24 ο ειπών τον ασεβή δίκαιός εστιν επικατάρατος λαοίς έσται και μισητός εις έθνη
25 οι δε ελέγχοντες βελτίους φανούνται επ' αυτούς δε ήξει ευλογία αγαθή
26 χείλη δε φιλήσουσιν αποκρινόμενα λόγους αγαθούς
27 ετοίμαζε εις την έξοδον τα έργα σου και παρασκευάζου εις τον αγρόν και πορεύου κατόπισθέν μου και ανοικοδομήσεις τον οίκόν σου
28 μη ίσθι ψευδής μάρτυς επί σον πολίτην μηδέ πλατύνου σοις χείλεσι
29 μη είπης ον τρόπον εχρήσατό μοι χρήσομαι αυτώ τίσομαι δε αυτόν α με ηδίκησεν
30 ώσπερ γεώργιον ανήρ άφρων και ώσπερ αμπελών άνθρωπος ενδεής φρενών
31 εάν αφής αυτόν χερσωθήσεται και χορτομανήσει όλος και γίνεται εκλελειμμένος οι δε φραγμοί των λίθων αυτού κατασκάπτονται
32 ύστερον εγώ μετενόησα επεβλέψα του εκλέξασθαι παιδείαν
33 ολίγον νυστάζω ολίγον δε καθυπνώ ολίγον δε εναγκαλίζομαι χερσί στήθη
34 εάν δε τούτο ποιής ήξει προπορευομένη η πενία σου και η ένδεια ώσπερ αγαθός δρομεύς
\par
25:1 αύται αι παροιμίαι Σολομώντος αι αδιάκριτοι ας εξεγράψαντο οι φίλοι Εζεκίου του βασιλέως της Ιουδαίας
2 δόξα θεού κρύπτει λόγον δόξα δε βασιλέως τιμά πράγματα
3 ουρανός υψηλός γη δε βαθεία καρδία δε βασιλέως ανεξέλεγκτος
4 τυπτε αδόκιμον αργύριον και καθαρίσθήσεται καθαρόν άπαν
5 κτείνε ασεβείς εκ προσώπου βασιλέως και κατορθώσει εν δικαιοσύνη ο θρόνος αυτού
6 μη αλαζονεύου ενώπιον βασιλέως μηδέ εν τόποις δυναστών υφίστασο
7 κρείσσον γαρ το ρηθήναί σοι ανάβαινε προς με η ταπεινώσαί σε εν προσώπω δυνάστου α είδον οι οφθαλμοί σου λέγε
8 μη πρόσπιπτε εις μάχην ταχέως ίνα μη μεταμεληθής επ' εσχάτων ηνίκα αν σε ονειδίση ο φίλος
9 αναχώρει εις τα οπίσω μη καταφρόνει
10 μη σε ονειδίση μεν ο φίλος η δε μάχη σου και η έχθρα ουκ απέσται αλλά έσται σοι ίση θανάτου χάρις και φιλία ελευθεροί ας τήρησον σεαυτώ ίνα μη επονείδιστος γένη αλλά φύλαξον τας οδούς σου ευσαναλλάκτως
11 μήλον χρυσούν εν ορμίσκω σαρδίου ούτως ειπείν λόγον
12 εις ενώτιον χρυσούν σάρδιον πολυτελές δέδεται λόγος σοφός εις ευήκοον ους
13 ώσπερ έξοδος χιόνος εν αμήτω κατά καύμα ωφελεί ούτως άγγελος πιστός τους αποστείλαντας αυτόν ψυχάς γαρ των αυτώ χρωμένων ωφελεί
14 ώσπερ άνεμοι και νέφη και υετοί επιφανέστατοι ούτως οι καυχώμενοι επί δόσει ψευδεί
15 εν μακροθυμία ευοδία βασιλεύσι γλώσσα δε μαλακή συντρίβει οστά
16 μέλι ευρών φάγε το ικανόν μή ποτε πλησθείς εξεμέσης
17 σπάνιον είσαγε σον πόδα προς σεαυτού φίλον μή ποτε πλησθείς σου μισήση σε
18 ρόπαλον και μάχαιρα και τόξευμα ακιδωτόν ούτως και ανήρ ο καταμαρτυρών του φίλου αυτού μαρτυρίαν ψευδή
19 οδός κακού και πους παρανόμου ολείται εν ημέρα κακή
20 ώσπερ όξος έλκει ασύμφορον ούτως προσπεσόν πάθος εν σώματι καρδίαν λυπεί ώσπερ σης ιματίω και σκώληξ ξύλω ούτως λύπη ανδρός βλάπτει καρδίαν
21 εάν πεινά ο εχθρός σου τρέφε αυτόν εάν διψά πότιζε αυτόν
22 τούτο γαρ ποιών άνθρακας πυρός σωρεύσεις επί της κεφαλής αυτού ο δε κύριος ανταποδώσει σοι αγαθά
23 άνεμος βορέας εξεγείρει νέφη πρόσωπον δε αναιδές γλώσσαν ερεθίζει
24 κρείσσον οικείν επί γωνίας δώματος η μετά γυναικός λοιδόρου εν οικία κοινή
25 ώσπερ ύδωρ ψυχρόν ψυχή διψώση προσηνές ούτως αγγελία αγαθή εκ γης μακρόθεν
26 ώσπερ ει τις πηγήν φράσσοι και ύδατος έξοδον λυμαίνοιτο ούτως άκοσμον δίκαιον πεπτωκέναι ενώπιον ασεβούς
27 εσθίειν μέλι πολύ ου καλόν τιμάν δε χρη λόγους ενδόξους
28 ώσπερ πόλις τα τείχη καταβεβλημένη και ατείχιστος ούτως ανήρ ος ου μετά βουλής τι πράσσει
\par
26:1 ώσπερ δρόσος εν αμητώ και ώσπερ υετός εν θέρει ούτως ουκ έστιν άφρονι τιμή
2 ώσπερ όρνεα πέταται και στρουθοί ούτως αρά ματαία ουκ επελεύσεται ουδενί
3 ώσπερ μάστιξ ίππω και κέντρον όνω ούτως ράβδος έθνει παρανόμω
4 μη αποκρίνου άφρονι κατά την εκείνου αφροσύνην ίνα μη όμοιος γένη αυτώ
5 αλλά αποκρίνου άφρονι προς την εκείνου αφροσύνην ίνα μη φαίνηται σοφός παρ' εαυτώ
6 εκ των οδών εαυτού όνειδος ποιείται ο αποστείλας δι' αγγέλου άφρονος λόγον
7 αφελού πορείαν σκελών και παροιμίας εκ στόματος αφρόνων
8 ος αποδεσμεύει λίθον εν σφενδόνη όμοιός εστι τω διδόντι άφρονι δόξαν
9 άκανθαι φύονται εν χειρί του μεθύσου δουλεία δε εν χειρί των αφρόνων
10 πολλά χειμάζεται πάσα σαρξ αφρόνων συντρίβεται γαρ η έκστασις αυτών
11 ώσπερ κύων όταν επέλθη επί τον εαυτού έμετον και μισητός γένηται ούτως άφρων τη εαυτού κακία αναστρέψας επί την εαυτού αμαρτίαν
12 είδον άνδρα δόξαντα παρ' εαυτώ σοφόν είναι ελπίδα μέντοι έσχε μάλλον άφρων αυτού
13 λέγει οκνηρός αποστελλόμενος εις οδόν λέων εν ταις οδοίς εν δε ταις πλατείαις φονευταί
14 ώσπερ θύρα στρέφεται επί του στρόφιγγος ούτως οκνηρός επί της κλίνης αυτού
15 κρύψας οκνηρός την χείρα εν τω κόλπω αυτού ου δύναται επενεγκείν επί το στόμα
16 σοφώτερος εαυτώ οκνηρός φαίνεται του εν πλησμονή αποκομίζοντος αγγελίαν
17 ώσπερ ο κρατών κέρκου κυνός ούτως ο προεστώς αλλοτρίας κρίσεως
18 ώσπερ οι ιώμενοι προβάλλουσι λόγους εις ανθρώπους ο δε απαντήσας τω λόγω πρώτος υποσκελισθήσεται
19 ούτως πάντες οι ενεδρεύοντες τους εαυτών φίλους όταν δε φωραθώσι λέγουσιν ότι παίζων έπραξα
20 εν πολλοίς ξύλοις θάλλει πυρ όπου δε ουκ έστι δίθυμος ησυχάζει μάχη
21 εσχάρα άνθραξι και ξύλα πυρί ανήρ δε λοίδορος εις ταραχήν μάχης
22 λόγοι κερκώπων μαλακοί ούτοι δε τύπτουσιν εις ταμιεία σπλάγχνων
23 αργύριον διδόμενον μετά δόλου ώσπερ όστρακον ηγητέον χείλη λεία καρδίαν καλύπτει λυπηράν
24 χείλεσι πάντα επινεύει αποκλαιόμενος εχθρός εν δε τη καρδία τεκταίνεται δόλους
25 εάν σου δεήται ο εχθρός μεγάλη τη φωνή μη πεισθής αυτώ επτά γαρ εισι πονηρίαι εν τη καρδία αυτού
26 ο κρύπτων έχθραν συνίστησι δόλον εκκαλύπτει δε τας εαυτού αμαρτίας εύγνωστος εν συνεδρίω
27 ο ορύσσων βόθρον τω πλησίον εμπεσείται εις αυτόν ο δε κυλίων λίθον εφ' εαυτόν κυλίει
28 γλώσσα ψευδής μισεί αλήθειαν στόμα δε άστεγον ποιεί ακαταστασίαν
\par
27:1 μη καυχώ τα εις αύριον ου γαρ γινώσκεις τι τέξεται η επιούσα
2 εγκωμιαζέτω σε ο πέλας και μη το σον στόμα αλλότριος και μη τα σα χείλη
3 βαρύ λίθος και δυσβάστακτον άμμος οργή δε άφρονος βαρυτέρα αμφοτέρων
4 ανελεήμων θυμός και οξεία οργή αλλ' ουδέν υφίσταται ζήλος
5 κρείσσους έλεγχοι αποκαλυπτόμενοι κρυπτομένης φιλίας
6 αξιοπιστότερα τραύματα φίλου η εκούσια φιλήματα εχθρού
7 ψυχή εν πλησμονή ούσα κηρίοις εμπαίζει ψυχή δε ενδεεί και τα πικρά γλυκέα φαίνεται
8 ώσπερ όρνεον όταν καταπετασθή εκ της ιδίας νοσσιάς ούτως άνθρωπος δουλούται όταν αποξενωθή εκ των ιδίων τόπων
9 μύροις και οίνοις και θυμιάμασι τέρπεται καρδία καταρρήγνυται δε υπό συμπτωμάτων ψυχή
10 φίλον σον η φίλον πατρώον μη εγκαταλίπης εις δε τον οίκον του αδελφού σου μη εισέλθης ατυχών κρείσσων φίλος εγγύς η αδελφός μακράν οικών
11 σοφός γίνου υιέ ίνα ευφραίνηταί μου η καρδία και απόστρεψον από σου επονειδίστους λόγους
12 πανούργος κακών επερχομένων απεκρύβη άφρονες δε επελθόντες ζημίαν τίσουσιν
13 αφελού το ιμάτιον αυτού παρήλθε γαρ υβριστής όστις τα αλλότρια λυμαίνεται
14 ος αν ευλογεί φίλον τοπρωϊ μεγάλη τη φωνή καταρωμένου ουδέν διαφέρειν δόξει
15 σταγόνες εκβάλλουσιν άνθρωπον εν ημέρα χειμερινή εκ του οίκου αυτού ωσαύτως και γυνή λοίδορος εκ του ιδίου οίκου
16 βορέας σκληρός άνεμος ονόματι δε επιδέξιος καλείται
17 σίδηρος σίδηρον οξύνει ανήρ δε παροξύνει πρόσωπον εταίρου
18 ος φυτεύει συκήν φάγεται τους καρπούς αυτής ος δε φυλάσσει τον εαυτού κύριον τιμηθήσεται
19 ώσπερ ουχ όμοια πρόσωπα προσώποις ούτως ουδέ αι καρδίαι των ανθρώπων όμοιαι
20 άδης και απώλεια ουκ εμπίμπλανται ωσαύτως και οι οφθαλμοί των ανθρώπων άπληστοι βδέλυγμα κυρίω στηρίζων οφθαλμόν και οι απαίδευτοι ακρατείς γλώσση
21 δοκίμιον αργυρίω και χρυσίω πύρωσις ανήρ δε δοκιμάζεται διά στόματος εγκωμιαζόντων αυτόν
22 εάν μαστιγοίς άφρονα εν μέσω συνεδρίω ατιμάζων ου μη περιέλης την αφροσύνην αυτού
23 γνωστώς επιγνώση ψυχάς ποιμνίου σου και επιστήσεις καρδίαν σου σαις αγέλαις
24 ότι ουκ εις τον αιώνα ανδρί κράτος και ισχύς ουδέ παραδίδωσιν εκ γενεάς εις γενεάν
25 επιμελού των εν τω πεδίω χλωρών και κερείς πόαν και συνάγαγε χόρτον ορεινόν
26 ίνα έχης πρόβατα εις ιματισμόν τίμα πεδίον ίνα ωσί σοι άρνες
27 υιέ παρ' εμού έχεις ρήσεις ισχυράς εις την ζωήν σου και εις την ζωήν σων θεραπόντων
\par
28:1 φεύγει ασεβής μηδενός διώκοντος δίκαιος δε ώσπερ λέων πέποιθε
2 δι' αμαρτίας ασεβών κρίσεις εγείρονται ανήρ δε πανούργος κατασβέσει αυτάς
3 ανδρείος εν ασεβείαις συκοφαντεί πτωχούς ώσπερ υετός λάβρος και ανωφωλής
4 ούτως οι εγκαταλειπόντες τον νόμον εγκωμιάζουσιν ασέβειαν οι δε αγαπώντες τον νόμον περιβάλλουσιν εαυτοίς τείχος
5 άνδρες κακοί ου νοήσουσι κρίμα οι δε ζητούντες τον κύριον συνήσουσιν εν παντί
6 κρείσσων πτωχός πορευόμενος εν αλήθεια πλουσίου ψευδούς
7 φυλάσσει νόμον υιός συνετός ος δε ποιμαίνει ασωτίαν ατιμάζει πατέρα αυτού
8 ο πληθύνων τον πλούτον αυτού μετά τόκων και πλεονασμών τω ελεούντι πτωχούς συνάγει αυτόν
9 ο εκκλίνων το ους αυτού του μη ακούσαι νόμου και αυτός την προσευχήν αυτού εβδέλυκται
10 ος πλανά ευθείς εν οδώ κακή εις διαφθοράν αυτός εμπεσείται οι δε άνομοι διελεύσονται αγαθά και ουκ εισελεύσονται εις αυτά
11 σοφός παρ' εαυτώ ανήρ πλούσιος πένης δε νοήμων καταγνώσεται αυτού
12 διά βοήθειαν δικαίων πολλή γίνεται δόξα εν δε τόποις ασεβών αλίσκονται άνθρωποι
13 ο επικαλύπτων ασέβειαν αυτού ουκ ευοδωθήσεται ο δε εξηγούμενος και ελέγχων αγαπηθήσεται
14 μακάριος ανήρ ος καταπτήσσει πάντα δι' ευλάβειαν ο δε σκληρός την καρδίαν εμπεσείται κακοίς
15 λέων πεινών και λύκος διψών ος τυραννεί πτωχός ων έθνους πενιχρού
16 βασιλεύς ενδεής προσόδων μέγας συκοφάντης ο δε μισών αδικίαν μακρόν χρόνον ζήσεται
17 άνδρα τον εν αιτία φόνου ο εγγυώμενος φυγάς έσται και ουκ εν ασφαλεία
18 ο πορευόμενος δικαιώς βεβοήθηται ο δε σκολιαίς οδοίς πορευόμενος εμπλακήσεται
19 ο εργαζόμενος την εαυτού γην πλησθήσεται άρτων ο δε διώκων σχολήν πλησθήσεται πενίας
20 ανήρ αξιόπιστος πολλά ευλογηθήσεται ο δε κακός ουκ ατιμώρητος έσται
21 ος ουκ αισχύνεται πρόσωπα δικαίων ουκ αγαθός ο τοιούτος ψωμού άρτου αποδώσεται άνδρα
22 σπεύδει πλουτείν ανήρ βάσκανος και ουκ οίδεν ότι ελεήμων κρατήσει αυτού
23 ο ελέγχων ανθρώπου οδούς χάριτας έξει μάλλον του γλωσσοχαριτούντος
24 ος αποβάλλεται πατέρα η μητέρα και δοκεί μη αμαρτάνειν ούτος κοινωνός εστιν ανδρός ασεβούς
25 άπληστος ανήρ κρινεί εική ος δε πέποιθεν επί κύριον εν επιμελεία έσται
26 ος πέποιθε θρασεία καρδία ο τοιούτος άφρων ος δε πορεύεται σοφία σωθήσεται
27 ος δίδωσι πτωχοίς ουκ ενδεηθήσεται ος δε αποστρέφει τον οφθαλμόν αυτού εν πολλή απορία έσται
28 εν τόποις ασεβών στένουσι δίκαιοι εν δε τη εκείνων απωλεία πληθυνθήσονται δίκαιοι
\par
29:1 κρείσσων ανήρ ελέγχων ανδρός σκληροτραχήλου εξαπίνης γαρ φλεγομένου αυτού ουκ έστιν ίασις
2 εγκωμιαζομένων δικαίων ευφρανθήσονται λαοί αρχόντων δε ασεβών στένουσιν άνδρες
3 ανδρός φιλούντος σοφίαν ευφραίνεται ο πατήρ αυτού ος δε ποιμαίνει πόρνας απολεί πλούτον
4 βασιλεύς δίκαιος ανίστησι χώραν ανήρ δε παράνομος κατασκάπτει
5 ος παρασκευάζεται επί πρόσωπον του εαυτού φίλου δίκτυον περιβαλλεί αυτό τοις εαυτού ποσίν
6 αμαρτάνοντι ανδρί μεγάλη παγίς δίκαιος δε εν χαρά και εν ευφροσύνη έσται
7 επίσταται δίκαιος κρίνειν πενιχροίς ο δε ασεβής ου νοεί γνώσιν και πτωχώ ουχ υπάρχει νους επιγνώμων
8 άνδρες λοιμοί εξέκαυσαν πόλιν σοφοί δε απέστρεψαν οργήν
9 ανήρ σοφός κρινεί έθνη ανήρ δε φαύλος οργιζόμενος καταγελάται και ου καταπτήσσει
10 άνδρες αιμάτων μέτοχοι μισήσουσιν όσιον οι δε ευθείς εκζητήσουσι ψυχήν αυτού
11 όλον τον θυμόν αυτού εκφέρει ο άφρων σοφός δε ταμιεύεται κατά μέρος
12 βασιλέως υπακούοντος λόγον άδικον πάντες οι υπ' αυτόν παράνομοι
13 δανειστού και χρεωφειλέτου αλλήλοις συνελθόντων επισκοπήν ποιείται αμφοτέρων ο κύριος
14 βασιλέως εν αλήθεια κρίνοντος πτωχούς ο θρόνος αυτού εις μαρτύριον κατασταθήσεται
15 πληγαί και έλεγχοι διδόασι σοφίαν παις δε πλανώμενος αισχύνει γονείς αυτού
16 πολλών όντων ασεβών πολλαί γίνονται αμαρτίαι οι δε δίκαιοι εκείνων πιπτόντων κατάφοβοι γίνονται
17 παίδευε υιόν σου και αναπαύσει σε και δώσει κόσμον τη ψυχή σου
18 ου μη υπάρξη εξηγητής έθνει παρανόμω ο δε φυλάσσων τον νόμον μακάριστος
19 λόγοις ου παιδευθήσεται οικέτης σκληρός εάν γαρ και νοήση ουχ υπακούσεται
20 εάν ίδης άνδρα ταχύν εν λόγοις γίνωσκε ότι ελπίδα έχει μάλλον ο άφρων αυτού
21 ος κατασπαταλά εκ παιδός οικέτης έσται έσχατον δε οδυνηθήσεται εφ' εαυτώ
22 ανήρ θυμώδης ορύσσει νείκος ανήρ δε οργίλος εξώρυξεν αμαρτίας
23 ύβρις άνδρα ταπεινοί τους δε ταπεινόφρονας ερείδει δόξη κύριος
24 ος μερίζεται κλέπτη μισεί την εαυτού ψυχήν εάν δε όρκου προτεθέντος ακούσαντες μη αναγγείλωσι
25 φοβηθέντες και αισχυνθέντες ανθρώπους υπεσκελισθήσονται ο δε πεποιθώς επί κύριον ευφρανθήσεται ασέβεια ανδρί δίδωσι σφάλμα ο δε πεποιθώς επί τω δεσπότη σωθήσεται
26 πολλοί θεραπεύουσι πρόσωπα ηγουμένων παρά δε κυρίου γίνεται το δίκαιον ανδρί
27 βδέλυγμα δίκαιος ανήρ ανδρί αδίκω βδέλυγμα δε ανόμω κατευθύνουσα οδός
\par
30:1 τάδε λέγει ο ανήρ τοις πιστεύουσι θεώ και παύομαι
2 αφρονέστατος γαρ ειμι απάντων ανθρώπων και φρόνησις ανθρώπου ουκ έστιν εν εμοί
3 ο θεός δεδίδαχέ με σοφίαν και γνώσιν αγίων έγνωκα
4 τις ανέβη εις τον ουρανόν και κατέβη τις συνήγαγεν ανέμους εν κόλπω τις συνέστρεψεν ύδωρ εν ιματίω τις εκράτησε πάντων των άκρων της γης τι όνομα αυτώ η τι όνομα τοις τέκνοις αυτού ίνα γνως
5 πάντες γαρ λόγοι θεού πεπυρωμένοι υπερασπίζει δε αυτός των ευλαβουμένων αυτόν
6 μη προσθής τοις λόγοις αυτού ίνα μη ελέγξη σε και ψευδής γένη
7 δύο αιτούμαι παρά σου μη αφέλης μου χάριν προ του αποθανείν με
8 μάταιον λόγον και ψευδή μακράν μου ποίησον πλούτον δε και πενίαν μη μοι δως σύνταξον δε μοι τα δέοντα και τα αυτάρκη
9 ίνα μη πλησθείς ψευδής γένωμαι και είπω τις με ορά η πενηθείς κλέψω και ομόσω το όνομα του θεού
10 μη παραδώς δούλον εις χείρας δεσπότου μή ποτε καταράσηταί σε και αφανισθής
11 έκγονον κακόν πατέρα καταράται της δε μητέρα ουκ ευλογεί
12 έκγονον κακόν δίκαιον εαυτόν κρινεί την δε έξοδον αυτού ουκ απένιψεν
13 έκγονον κακόν υψηλούς οφθαλμούς έχει τοις δε βλεφάροις αυτού επαίρεται
14 έκγονον κακόν μαχαίρας οδόντας έχει και τας μύλας τομίδας ώστε αναλίσκειν τους ασθενείς από της γης και τους πένητας αυτών εξ ανθρώπων
15 τη βδέλλη τρεις θυγατέρες ήσαν αγαπήσει αγαπώμεναι αι τρεις αύται ουκ ενεπίμπλασαν αυτήν και η τετάρτη ουκ ηρκέσθη ειπείν ικανόν
16 άδης και έρως γυναικός και γη ουκ εμπιπλαμένη ύδατος και ύδωρ και πυρ ου μη είπωσιν αρκεί
17 οφθαλμόν καταγελώντα πατρός και ατιμάζοντα γήρας μητρός εκκόψαισαν αυτόν κόρακες εκ των φαράγγων και καταφάγοισαν αυτόν νεοσσοί αετών
18 τρία δε εστιν αδύνατά μοι νοήσαι και το τέταρτον ουκ επιγινώσκω
19 ίχνη αετού πετομένου και οδούς όφεως επί πέτρας και τρίβους νηός ποντοπορούσης και οδούς ανδρός εν νεότητι
20 τοιαύτη οδός γυναικός μοιχαλίδος η όταν πράξη απονιψαμένη ουδέν φησι πεπραχέναι άτοπον
21 διά τριών σείεται η γη το δε τέταρτον ου δύναται φέρειν
22 εάν οικέτης βασιλεύση και άφρων πλησθή σιτίων
23 και οικέτις εάν εκβάλη την εαυτής κυρίαν και μισητή γυνή εάν τύχη ανδρός αγαθού
24 τέσσαρα δε εστιν ελάχιστα επί της γης ταύτα δε εστι σοφώτερα των σοφών
25 οι μύρμηκες οις μη εστιν ισχύς και ετοιμάζονται θέρους την τροφήν
26 και οι χοιρογρύλλιοι έθνος ουκ ισχυρόν οι εποιήσαντο εν πέτραις τους εαυτών οίκους
27 αβασίλευτόν εστιν η ακρίς και εκστρατεύει αφ' ενός κελεύσματος ευτάκτως
28 και καλαβώτης χερσίν ερειδόμενος και ευάλωτος ων κατοικεί εν οχυρώμασι βασιλέων
29 τρία δε εστιν α ευόδως πορεύεται και το τέταρτον ο καλώς διαβαίνει
30 σκύμνος λέοντος ισχυρότερος κτηνών ος ουκ αποστρέφεται ουδέ καταπτήσσει κτήνος
31 και αλέκτωρ εμπεριπατών εν θηλείαις ευψύχως και τράγος ηγούμενος αιπολίου και βασιλεύς δημηγορών εν έθνει
32 εάν πρόη σεαυτόν εις ευφροσύνη και εκτείνης την χείρά σου μετά μάχης ατιμασθήση
33 άμελγε γάλα και έσται βούτυρον εάν δε εκπιέζης μυκτήρας εξελεύσεται αίμα εάν δε εξέλκης λόγους εξελεύσονται κρίσεις και μάχαι
\par
31:1 οι εμοί λόγοι είρηνται υπό θεού βασιλέως χρηματισμός ον επαίδευσεν η μήτηρ αυτού
2 τι τέκνον τηρήσεις τι ρήσεις θεού πρωτογενές σοι λεγω υιέ τι τέκνον εμής κοιλίας τι τέκνον εμών ευχών
3 μη δως γυναιξί σον πλούτον και τον σον νουν και βίον εις υστεροβουλίαν μετά βουλής πάντα ποίει μετά βουλής οινοπότει
4 οι δυνάσται θυμώδεις εισίν οίνον μη πινέτωσαν
5 ίνα μη πίνοντες επιλάθωνται της σοφίας και ορθά κρινείν ου μη δύνωνται τους ασθενείς
6 δίδοτε μέθην τοις εν λύπαις και οίνον πίνειν τοις εν οδύναις
7 ίνα επιλάθωνται της πενίας και των πόνων μη μνησθώσιν έτι
8 υιέ άνοιγε σον στόμα λόγω θεού και κρίνε πάντας υγιώς
9 άνοιγε σον στόμα και κρίνε δικαιώς διάκρινε δε πένητα και ασθενή
10 γυναίκα ανδρείαν τις ευρήσει τιμιωτέρα δε εστι λίθων πολυτελών η τοιαύτη
11 θαρσεί επ' αυτή η καρδία του ανδρός αυτής η τοιαύτη καλών σκύλων ουκ απορήσει
12 ενεργεί γαρ τω ανδρί αγαθόν και ου κακόν πάντα τον βίον
13 μηρυομένη έρια και λίνον εποίησεν εύχρηστα ταις χερσίν αυτής
14 εγένετο ωσεί ναυς εμπορευομένη μακρόθεν συνάγει αυτής τον πλούτον
15 και ανίσταται εκ νυκτών και έδωκε βρώματα τω οίκω και έργα ταις θεραπαίναις
16 θεωρήσασα γεώργιον επρίατο από δε καρπών χειρών αυτής κατεφύτευσε κτήμα
17 αναζωσαμένη ισχυρώς την οσφύν αυτής ήρεισε τους βραχίονας αυτής εις έργον
18 εγεύσατο ότι καλόν εστι το εργάζεσθαι και ουκ αποσβέννυται ο λύχνος αυτής όλην την νύκτα
19 τους πήχεις  αυτής εκτείνει επί τα συμφέροντα τας δε χείρας αυτής ερείδει εις άτρακτον
20 χείρας δε αυτής διήνοιξε πένητι καρπόν δε εξέτεινε πτωχώ
21 ου φροντίζει των εν οίκω ο ανήρ αυτής όταν που χρονίζη πάντες γαρ οι παρ' αυτής ενδεδυμένοι εισί
22 δισσάς χλαίνας εποίησε τω ανδρί αυτής εκ δε βύσσου και πορφύρας εαυτή ενδύματα
23 περίβλεπτος δε γίνεται ο ανήρ αυτής εν πύλαις ηνίκα αν καθίση εν συνεδρίω μετά των γερόντων της γης
24 σινδόνας εποίησε και απέδοτο  περιζώματα τοις Χαναναίοις
25 ισχύν και ευπρέπειαν ενεδύσατο και ευφράνθη εν ημέραις εσχάταις
26 στόμα δε αυτής ανοίγει σοφώς και νομοθέσμως η δε ελεημοσυνη αυτης εν τη γλωσση αυτης
27 στεγναί διατριβαί οίκω αυτής σίτα δε οκνηρά ουκ έφαγε
28 ανέστησε τα τέκνα αυτής και επλούτησαν και ο ανήρ αυτής ήνεσεν αυτήν
29 πολλαί θυγατέρες εκτήσαντο πλούτον πολλαί εποίησαν δύναμιν συ δε υπέρκεισαι υπερήρας πάσας
30 ψευδείς αρέσκειαι και μάταιον κάλλος ουκ έστιν εν σοι γυνή γαρ συνετή ευλογείται φόβον δε κυρίου αυτή αινείτω
31 δότε αυτή από καρπών χειρών αυτής και αινείσθω εν πύλαις ο ανήρ αυτής

\end{document}

